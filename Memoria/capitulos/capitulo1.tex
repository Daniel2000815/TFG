% !TeX root = ../libro.tex
% !TeX encoding = utf8
\chapter{Fundamentos matemáticos: las SDFs}
Estamos acostumbrados a representar superficies en $\R^3$ a través de ecuaciones paramétricas e implícitas. En el caso de las paramétricas se asigna a cada tupla de parámetros un punto, mientras que para las implícitas, dado un punto la ecuación indica si este se encuentra dentro o fuera de la superficie. El tipo de superficies implícitas más comúnmente estudiado es el de las superficies algebraicas, variedades algebraicas de dimensión 2. Existen varios métodos para la visualización de este tipo de superficies. Uno de ellos es tratar de generar una malla de polígonos previamente a partir de la ecuación para después ser visualizada en tiempo real usando los métodos clásicos. El problema de este método es que no siempre se puede aplicar y conlleva pérdida de precisión en la representación de la superficie. Otro método es el \textit{raytracing}, pero este también puede llegar a perder precisión, además de que es muy lento, haciendo que la representación de una superficie tan simple como una esfera sea computacionalmente muy costoso.\newline

Como solución a esto, T. Sederberg y A.Zundel \cite{Sederberg1989ScanLD} presentaron en 1989 un método capaz de representar superficies algebraicas sin pérdida de información y de manera eficiente, capaz además de trabajar con siluetas, intersección de curvas y operaciones booleanas. En 1889 John C. Hart \cite{hart2} presenta una técnica de \textit{raymarching} para la representación de fractales usando funciones distancia con signo. Posteriormente, en 1995 \cite{hart} generaliza esta técnica con el uso de \textit{spheretracing} para la representación de cualquier superficie implícita (algebraica o no), punto de partida de este trabajo.\newline

Este método nos permitirá representar cualquier superficie en tiempo real con un coste computacionalmente muy bajo y a cualquier nivel de detalle. No obstante, para usarlo debemos comprender qué son las funciones distancia con signo, sus propiedades, y como trabajar con ellas.

\begin{definicion}\label{def:sdf}
  Sea $\Omega \subset \R^3$. La \textbf{función distancia} asociada a $\Omega$, que llamamos $d_{\Omega}$ es el campo escalar que a cada punto de $\R^3$ le asigna su menor distancia a la frontera de $\Omega$:
    \begin{align*}
          d_{\Omega} \colon \R^3 &\to \R_0^+,\\
          x &\mapsto \inf\{\Vert x-y\Vert) \colon y \in \delta\Omega\}.
    \end{align*}

    Cuando $\Omega$ sea cerrado, podremos usar el mínimo en lugar del ínfimo.
\end{definicion}

\begin{definicion}\label{d:sdf}
  Sea $\Omega \subset \R^3$. La \textbf{función distancia con signo} asociada a $\Omega$ es el campo escalar de la forma:
  \begin{align*}
          \phi_{\Omega} \colon \R^3 &\to \R,\\
          x &\mapsto \begin{cases}
      d_{\Omega}(x),  &x\in \R^3\setminus \mathring{\Omega}, \\
      -d_{\Omega}(x), &x\in \mathring{\Omega}.
    \end{cases}
    \end{align*}

  En general, nos referiremos a esta función por sus siglas en inglés SDF (\textit{Signed Distance Function}), y la denotaremos simplemente $\phi$ siempre que no haya confusión.
\end{definicion}

 \begin{observacion}
     Un campo escalar $f\colon \R^3\to \R$ cualquiera será una función distancia si existe al menos un $\Omega \subset \R^3$ tal que $f = d_{\Omega}$. De la misma forma, $f$ será una SDF cuando para dicho $\Omega$ se tenga $f=\phi_{\Omega}$.
 \end{observacion}

\begin{definicion}
  Dada una función $\phi\colon \R^3 \to \R$ y $k\in\R$, llamamos \textbf{isosuperficie de $\boldsymbol{\phi}$ con valor $\boldsymbol{k}$} al conjunto:
  \begin{equation*}
      S_{\phi,k} = \{(x,y,z) :  \phi(x,y,z)=k\}.
  \end{equation*}
  Sin pérdida de generalidad podemos suponer $k=0$, pues de no ser el caso, tomamos la función $\phi'(x,y,z)=\phi(x,y,z)-k$ y tenemos que $S_{\phi',0} = S_{\phi,k}$. Por tanto, la denotaremos como $S_\phi$.
\end{definicion}

Nuestra intención será entonces construir una escena definida como la isosuperficie generada por un SDF. A partir de ahora, tomaremos $p=(x,y,z)\in\R^3$.

\begin{ejemplo}\label{ej:sdf}
    Ejemplos simples de SDF $\phi$ en $p$ para diferentes conjuntos $\Omega$ son:
    \begin{itemize}
        \item \textbf{Esfera de radio $\boldsymbol{r}$ centrada en el origen.}
        \begin{equation*}
            \Omega=\{x\in \R^3 : \Vert x\Vert = r\},\quad \phi(p) = \Vert p\Vert - r.
        \end{equation*}
        \item \textbf{Plano con vector normal unitario $\boldsymbol{n=(a,b,c)}$ y pasando por el origen}.
        \begin{equation*}
            \Omega=\{p\in\R^3 : ax+by+cz = 0\},\quad \phi(p) = p\cdot n.
        \end{equation*}
        \item \textbf{Toro sobre el eje Y de radios $\boldsymbol{R}$ y $\boldsymbol{r}$, con $\boldsymbol{R>r}$}:
        \begin{equation*}
            \Omega=\left\{p\in \R^3 : \left(R-\sqrt{x^2+z^2}\right)^2 + y^2 = r^2\right\},\quad \phi(p)= \bigg\Vert (\Vert(x,0,z)\Vert - R, y) \bigg\Vert - r.
        \end{equation*}
    \end{itemize}
\end{ejemplo}

\section{Diferenciabilidad}
Antes de seguir avanzando vamos a realizar un estudio de la diferenciabilidad de las funciones distancia con signo, pues nos será de utilidad en las siguientes secciones. Empezamos recordando varios conceptos de análisis diferencial \cite{diff} fijadas las variables $\{x_1,x_2,x_3\}$ y la base usual $$B=\{e_1,e_2,e_3\} = \{(1,0,0),(0,1,0),(0,0,1)\}.$$

Cuando sea conveniente identificaremos
\begin{equation*}
    x_1=x,\quad x_2=y,\quad x_3 = z.
\end{equation*}

\begin{definicion}\label{def:parcial}
    Sea $U$ un abierto de $\R^3$ y $\phi: U \to \R$. Para $i\in \{1,2,3\}$, definimos la \textbf{$\boldsymbol{i}$-ésima derivada parcial} de $\phi$ en $p_0\in\R^3$ como
    \begin{equation*}
        \frac{\partial \phi}{\partial x_i}(p_0) = \lim_{h\to 0}\frac{\phi(p_0+he_i) - \phi(p_0)}{h}.
    \end{equation*}
\end{definicion}

\begin{definicion}
    Sea $U$ un abierto de $\R^3$ y $\phi: U \to \R$. Diremos que $\phi$ es \textbf{diferenciable} en $p_0 \in U$ si existen todas sus derivadas parciales en $p_0$ y son continuas. Definimos la \textbf{diferencial} de $\phi$ en $p_0$ como la suma de todas sus parciales en dicho punto, y la denotamos como $d\phi(p_0)$.

    % \begin{equation*}
    %     \text{lím}_{p\to p_0} \frac{\phi(p)-\phi(p_0) - \nabla\phi(p_0)\cdot(p-p_0)}{\Vert p-p_0\Vert} = 0.
    % \end{equation*}
\end{definicion}

\begin{definicion}
    Dado un abierto $U\subseteq \R^3$, diremos que la \textbf{clase de diferenciabilidad} de una función $\phi:U\to \R$ es $\mathbb{C}^n(U)$ si para $i\in \{1,2,3\}$ y $j\in \{0,\dots, n\}$ existen y son continuas todas las parciales
    \begin{equation*}
        \frac{\partial^j \phi}{\partial x_i}(p),\ \text{para todo } p\in U.
    \end{equation*}
\end{definicion}

\begin{definicion}
  Llamamos \textbf{gradiente} de $\phi\colon \R^3\to\R$ a la función
  \begin{align*}
      \nabla\phi\colon \R^3 &\to \R^3,\\
      p &\mapsto \left(\frac{\partial \phi}{\partial x_1}(p), \frac{\partial \phi}{\partial x_2}(p), \frac{\partial \phi}{\partial x_3}(p)\right).
  \end{align*}
\end{definicion}

\begin{definicion}
  Dada $\phi:\R^3\to \R$ diferenciable, definimos la \textbf{derivada direccional} en $p_0\in \R^3$ en la dirección $v\in\R^3$ a:
  \begin{equation*}
    \nabla_v \phi(p_0) = \nabla \phi(p_0) \cdot v = \frac{\partial{\phi}(p_0)}{\partial{x}}v_x + \frac{\partial{\phi}(p_0)}{\partial{y}}v_y + \frac{\partial{\phi}(p_0)}{\partial{z}}v_z.
  \end{equation*}
\end{definicion}




Ahora mismo, dado una función distancia con signo arbitraria no tenemos información alguna sobre su diferenciabilidad, ya que su expresión puede ser de lo más variada y compleja. Veamos una propiedad que cumplen todos las funciones distancia con signo y que nos permitirá obtener algo de información al respecto \cite{lips,derivWiki}.

\begin{definicion}
    Una campo escalar $\phi\colon \R^3 \to \R$ se dice \textbf{lipschitziano} si existe una constante $L>0$ tal que
    \begin{equation*}
        \vert \phi(p)-\phi(q)\vert \le L\Vert p-q\Vert,\ \text{para todo } p,q\in \R^3.
    \end{equation*}
    La constante $L$ recibe el nombre de \textbf{constante de Lipschitz}.
\end{definicion}

\begin{proposicion}
    Sea $\phi\colon \R^3 \to \R$ una función lipschitziana cualquiera con constante de Lipschitz $L$. Entonces
    \begin{equation*}
        \vert d\phi(p)\vert \le L
    \end{equation*}
    en todo punto donde sea diferenciable.
\end{proposicion}

\begin{lema}
    Sea $\phi:\R^3\to \R$ la función distancia con signo asociada a $\Omega$. Entonces $\phi$ es lipschitziana con constante $L=1$.
\end{lema}
\begin{proof}
    Sean $p$ y $q\in \R^3$. Usando la \autoref{def:sdf}, para todo $s\in \delta\Omega$ se tiene
    \begin{equation*}
        \phi(p) \le \Vert p-s\Vert = \Vert p-q+q+s\Vert \le \Vert p-q\Vert + \Vert q-s\Vert.
    \end{equation*}
    Por tanto, $\phi_{\Omega}(p) - \Vert p-q\Vert \le \Vert q-s \Vert$, luego $\phi_{\Omega}(p) - \Vert p-q\Vert \le \Inf_{s\in \delta\Omega}(\Vert q-s \Vert) = \phi_{\Omega}(q)$ y obtenemos
    \begin{equation*}
         \phi_{\Omega}(p) - \phi_{\Omega}(q) \le \Vert p-q \Vert.
    \end{equation*}
    De forma análoga podemos ver que $\phi_{\Omega}(q) - \phi_{\Omega}(p) \le \Vert q-p \Vert$, concluyendo que
    \begin{equation*}
        \vert \phi_{\Omega}(p) - \phi_{\Omega}(q)\vert \le 1\cdot \Vert p-q\Vert.\qedhere
    \end{equation*}
\end{proof}

\begin{lema}[Teorema de Rademacher]
    Sea $U$ un abierto de $\R^3$ y $\phi:U\to \R$ lipschitziana. Entonces $\phi$ es diferenciable en casi todo punto de $U$.
\end{lema}

Tenemos por tanto asegurado que $\phi_{\Omega}$ será diferenciable en casi todo punto de $\R^3$. No obstante, podemos concretar aún más dónde están los puntos de conflicto cuando $\Omega$ sea lo suficientemente regular \cite{dif1,dif2}. Para ello necesitaremos introducir el concepto de esqueleto de una superficie \cite{derivWiki} y repasar algunas definiciones básicas asociadas a superficies en el espacio \cite{apuntes:curvas}.
\begin{definicion}
    Sea $\phi_{\Omega} \colon \R^3\to \R$ una función distancia con signo. Llamamos \textbf{esqueleto} de $\Omega$ al conjunto de puntos de $\R^3$ cuya distancia a la superficie puede obtenerse como la distancia a dos o más puntos distintos de $\delta \Omega$:
    \begin{equation}
        \epsilon(\Omega) = \{p\in \R^3 : \phi_{\Omega}(p) = \Vert p-q\Vert = \Vert p-r\Vert,\ q,r\in \delta\Omega ,\ q\neq r \}.
    \end{equation}
\end{definicion}

\begin{definicion}
    Dado $I\subseteq \R$, llamamos \textbf{curva parametrizada} a una aplicación
    \begin{align*}
        \alpha: I &\to \R^3,\\
        t &\mapsto (x(t), y(t), z(t)),
    \end{align*}
    donde $x,y,z:I\to \R$ son diferenciables. 
\end{definicion}
\begin{definicion}
    Decimos que $\Omega \subseteq \R^3$ es una \textbf{superficie regular} si para cada $p\in \Omega$ existen abiertos $U\subseteq \R^2$ y $V\subseteq \R^3$ junto a una aplicación $\psi\colon U \to V\cap \Omega$ tal que:
    \begin{enumerate}
        \item $\psi$ es un homeomorfismo, es decir, es continua, biyectiva y con inversa continua,
        \item $\psi$ es diferenciable y su diferencial es inyectiva.
    \end{enumerate}
\end{definicion}

\begin{definicion}
    Sea $\Omega$ una superficie regular y $p\in \Omega$. Dados $\varepsilon>0$ y una curva parametrizada diferenciable
    $$\alpha: ]\varepsilon, \varepsilon[ \to \R^3 \text{ tal que } \text{Img($\alpha$)} \subset \Omega \text{ y } \alpha(0)=p,$$
    diremos que $\alpha'(0)$ es un \textbf{vector tangente} a $\Omega$ en $p$.
\end{definicion}
\begin{definicion}
    Sea $\Omega$ una superficie regular, $p\in \Omega$ y $T_p\Omega$ el plano vectorial conteniendo todos los vectores tangentes a $\Omega$ en $p$. Llamamos \textbf{plano tangente} a $\Omega$ en $p$ al conjunto $p+T_p\Omega$.
\end{definicion}
\begin{definicion}
    Sea $\Omega$ una superficie regular y $p\in \Omega$. El \textbf{vector normal} a $\Omega$ en $p$ es el vector $N_p\in \R^3$ de norma uno perpendicular al plano tangente de $\Omega$ en $p$.
\end{definicion}
% \begin{definicion}
%     Sea $\Omega \subset \R^3$ y $p\in \Omega$. Llamamos \textbf{vector normal} en $p$ al vector de norma uno y perpendicular al borde de $\Omega$ en $p$. Lo denotamos $N_p$.
% \end{definicion}



El siguiente teorema, cuya demostración podemos consultar en \cite{dif1}, nos proporciona una caracterización geométrica de la diferenciabilidad de cualquier función distancia con signo $\phi_{\Omega}$ bajo ciertas hipótesis de regularidad para $\Omega$.
\begin{teorema}\label{teo:diff}
    Sea $\Omega \subseteq \R^3$ cuya frontera es regular y $\phi_{\Omega} \colon \R^3\to \R$ la función distancia con signo asociada a $\Omega$. Entonces $\phi_{\Omega}$ es diferenciable en un entorno tubular $U$ de $\delta \Omega$. Es más, para cada $p\in \R^3$ se cumple una se las siguientes propiedades:
    \begin{enumerate}
        \item $p\in \delta \Omega$ y $\phi_{\Omega}$ es diferenciable en $p$ con $\nabla S_{\phi(p)} = N_p$,
        \item $p\notin \delta \Omega$ y $\phi_{\Omega}$ es diferenciable en $p$ si y solo si $p\in \R^3\setminus \epsilon(\Omega)$, en cuyo caso
        \begin{equation*}
            \nabla \phi_{\Omega}(p) = \frac{q-p}{\phi_{\Omega}(p)},
        \end{equation*}
        donde $q$ es el único punto de $\delta\Omega$ tal que $\phi_{\Omega}(p) = \Vert q-p\Vert$.
    \end{enumerate}
\end{teorema}

\begin{corolario}
    Toda función distancia con signo $\phi\colon \R^3\to \R$ satisface la ecuación de la eikonal 
    \begin{equation*}
        \Vert \nabla \phi(p)\Vert = 1
    \end{equation*}
    en todo punto $p$ donde sea diferenciable.
\end{corolario}
% A la vista de este resultado podemos diferenciar $\phi$ con tranquilidad, pues solo estamos interesados en estudiar el gradiente en puntos de $\delta\Omega$, luego
% \begin{equation*}
    
% \end{equation*}

% TODO: ver si decir que podria calcularse fuera del shader pasanndo el string al shader y recompilando. Si no, ver si comentar algo de que hay que calcularlo cada frame, y por eso no podemos hacerlo analiticamente
% De ser derivable, podría ser una buena opción calcular el gradiente una única vez al momento de definir $\phi$, tras lo cual para obtener $N$ solo habría que realizar evaluaciones de dicho gradiente. Sin embargo, esto requeriría de varias comprobaciones previas, y aún necesitaríamos otro método para tratar las no diferenciables. Por ello nos decantaremos por un método numérico más sencillo que únicamente hará uso de evaluaciones de $\phi$.



\section{Funciones cota de distancia}
Hemos visto que las SDFs nos proporcionan la distancia signada exacta en cada punto al más cercano de un conjunto $\Omega\subset \R^3$, pero lo cierto es que para representar la isosuperficie generada por un SDF no se necesita tanta precisión, ya que mientras dos funciones tengan los mismos ceros generarán la misma isosuperficie. Sin embargo, para representar la superficie en las siguientes secciones usaremos el método del \textit{spheretracing} (\autoref{sec:tracing}), que sí utiliza la información de la distancia en puntos diferentes a la frontera de $\Omega$. Introducimos un concepto que nos permitirá seguir usando este método pero no es tan restrictivo como el de SDF \cite{hart}.

\begin{definicion}
    Sea $\Omega\subset \R^3$ y $\phi$ su SDF. Una \textbf{cota de la distancia con signo} asociada a $\Omega$ es un campo escalar $\gamma\colon \R^3\to \R$ cumpliendo
    \begin{equation*}
        \vert \gamma(p)\vert \le \vert \phi(p)\vert,\ \forall p\in\R^3.
    \end{equation*}

    y que tiene el mismo signo que $\phi$ en cada punto. Nos referiremos a estas funciones como SDB por sus siglas en inglés (Signed Distance Bound).
\end{definicion}

Observamos que una SDF es un caso especial de SDB en el que se cumple la igualdad. De esta definición es evidente que una SDF y una SDB asociadas a un mismo $\Omega$ tienen los mismos ceros, y generan por tanto la misma isosuperficie. Aunque trabajar con una SDB en \textit{spheretracing} hará que la convergencia sea más lenta, en ocasiones una SDB es más rápida de evaluar que una SDF por tener una expresión más simple, haciendo que sea deseable trabajar con ellas. Habrá otras ocasiones en las que incluso será imposible obtener una SDF para un cierto $\Omega$. Por estos motivos en la literatura no se suele distinguir entre SDF y SDB, y como mucho se utilizan los términos SDF exacta y aproximada cuando se quiere realizar una distinción.\newline

Veamos algunos ejemplos de SDF aproximadas.
\begin{ejemplo}\label{ej:sdf}
    Para los siguientes conjuntos $\Omega$ podemos definir las SDFs aproximadas:
    \begin{itemize}
        \item \textbf{Cono centrado en el origen a lo largo del eje Y con altura $\boldsymbol{h}$ y ángulo $\theta$.}
        \begin{align*}
            &\Omega=\{p\in \R^3 : (x^2+z^2)(\cos\theta)^2 - y^2(\sin\theta)^2\},\\[8pt]
            &\phi(p) = \Max\Big((\sin\theta, \cos\theta)\cdot (\Vert (x,0,z)\Vert, y), -h-y\Big).
        \end{align*}
        \item \textbf{Elipsoide}.
        \begin{align*}
            \Omega=\{p\in\R^3 : \frac{x^2}{a^2}+\frac{y^2}{b^2}+\frac{z^2}{c^2} = 0,\ a,b,c\in \R\setminus \{0\}\},\\[10pt]
            \phi(p) = \Big\Vert \left( \frac{x}{a},\frac{y}{b},\frac{z}{c} \right) \Big\Vert \cdot \left(  \Big\Vert \left( \frac{x}{a},\frac{y}{b},\frac{z}{c} \right) \Big\Vert -1\right)\ \Big{/}\ \Big\Vert \left( \frac{x}{a^2},\frac{y}{b^2},\frac{z}{c^2} \right) \Big\Vert .
        \end{align*}
    \end{itemize}
\end{ejemplo}


\section{Operaciones sobre SDF}\label{sec:operaciones}

Si bien estas primitivas son fáciles de generar, también son muy simples y nos serán insuficientes si queremos construir escenas más complejas. Como comentamos en la introducción, una de las principales ventajas del uso de ecuaciones implícitas para representar modelos geométricos es la facilidad de combinación de estas primitivas, por ejemplo mediante operaciones booleanas o deformaciones. Sin embargo los sistemas que hacían uso de esta técnica no estaban lo suficientemente estructurados como para permitir aplicar estas operaciones de manera general e intuitiva, haciendo que no se pudieran aplicar de forma local y por tanto no se pudieran generar escenas complejas.\newline


Esto cambió en 1999 con el artículo de B.Wyvill y otros \cite{blobtree}, en el que sugieren usar una estructura de árbol para definir modelos como combinación de otros a través de operaciones básicas. La gran ventaja de este método es que es muy extensible, y además permite ver de forma muy clara la estructura del modelo. En esta sección estudiaremos los principales tipos de estas operaciones. En el \autoref{ap:operacionesSDF} podemos verlas en acción sobre diferentes primitivas.

\subsection{Operaciones booleanas}
Una de las técnicas más útiles para generar nuevas formas a partir de primitivas es la geometría de sólidos constructiva. Por la naturaleza de los SDF, estas operaciones se implementan fácilmente usando las funciones $\Max$ y $\Min$.

\begin{definicion}[Operaciones Booleanas]\label{p:boolean}
    Sean $A$ y $B$ isosuperficies generadas por $\phi$ y $\gamma$ respectivamente. Definimos los SDF para las siguientes operaciones.
    \begin{itemize}
        \item \textbf{Unión: } $\sdf_{A\cup B}(p) = \Min(\phi(p), \gamma(p))$,
        \item \textbf{Intersección: } $\sdf_{A\cap B}(p) = \Max(\phi(p), \gamma(p))$,
        \item \textbf{Diferencia: } $\sdf_{A\setminus B}(p) = \Max(\phi(p), -\gamma(p))$.
    \end{itemize}
\end{definicion}

\begin{observacion}
    Solo en el caso de la unión se obtiene un SDF exacto, ya que al aplicar $\Max$ en el interior de la superficie (donde $\phi(p) < 0$) el resultado puede ser solo una cota inferior de la distancia. En nuestro caso solo estamos interesados en visualizar la frontera de las superficies así que podemos obviar este problema, con la salvedad de que el algoritmo de \textit{raymarching} requiera de más iteraciones.
\end{observacion}

Un problema de usar estas transformaciones es que produce discontinuidades en la derivada del SDF resultante. Trataremos de evitar esta situación, además de por motivos analíticos, por motivos visuales, ya que esto produce bordes muy acusados en la intersección de ambas superficies. Existen muchas formas de combinar SDF de forma más natural. Usaremos una de las más extendidas, usada por programas de modelado 3D como Blender \cite{repo:blender} o videojuegos como Dreams \cite{game:dreams}, y que ha sido estudiada por Íñigo Quílez en su web \cite{article:smooth}.

\begin{observacion}
    Para mayor claridad del razonamiento, en las figuras se representarán funciones de variable real, a pesar de que nosotros trabajamos en $\R^3$.
\end{observacion}

Explicaremos la técnica poniendo como ejemplo la unión, y al final veremos como la intersección y la diferencia se deducen fácilmente de esta. La idea es, dadas $\phi$ y $\gamma$, añadir una corrección para cada punto a la función $\Min$ original para que cumpla ciertos requisitos. Por comodidad, definiremos 
\begin{align*}
      \Min_{\phi,\gamma}\colon \R^3 &\to \R,\\
      p &\mapsto \Min(\phi(p),\gamma(p)),
\end{align*}
y siguiendo un abuso de notación, escribiremos $\Min(p)$ cuando no haya lugar a dudas.\newline

Llamaremos a la mencionada corrección $\omega_k\colon \R^3\to\R$, donde $k\in \R_0^+$ es un coeficiente que controlará la intensidad del suavizado. Por tanto, la versión suavizada de la función $\Min$ original será
\begin{align*}
      \smin_{\phi,\gamma}\colon \R^3 &\to \R,\\
      p &\mapsto \Min_{\phi,\gamma}(p) - \omega_k(p).
\end{align*}
Siempre que no haya confusión, denotaremos $\smin_{\phi,\gamma} = \smin$.\newline

Como no queremos que este cambio afecte al algoritmo de \textit{raymarching}, debemos asegurar que se cumpla $\Min(p)\ge \smin(p)$, esto es,
\begin{equation*}
\omega_k(p)\ge 0,\ \forall p\in \R^3,\ \forall k\in \R^+_0.
\end{equation*}

Si estudiamos como se comporta la versión real de $\Min$ en la \autoref{fig:min_real}, vemos que los puntos de conflicto se encuentran cerca de las intersecciones de $\phi$ y $\gamma$, es decir, cuando $\phi$ y $\gamma$ están arbitrariamente cerca. En el resto de puntos no queremos modificar la función original, luego estudiaremos el comportamiento de $\smin$ en el conjunto de entornos de las intersecciones. Usaremos el valor de $k$ para decidir el tamaño de estos entornos, aplicando la corrección únicamente en los puntos del conjunto
\begin{equation*}
    B_{k} = \{p\in\R^3 : |\phi(p)-\gamma(p)| \le k\},
\end{equation*}

de forma que $\omega(p) = 0$ cuando $p\notin B_{k}$.\newline

\begin{figure}[t]
    \centering
    \includegraphics[width=0.75\textwidth]{Plantilla-TFG-master/img/smooth_real.png}
    \caption{Gráfica de $\Min\colon \R\to\R$}
    \label{fig:min_real}
\end{figure}

Para asegurar que $\smin$ sea continua en la frontera de $B_{k}$, imponemos la condición 

\begin{equation*}
    \omega_k(p) = 0,\ \forall p \in \delta B_{k}.
\end{equation*}

Por otro lado, es lógico que $\omega$ tenga su mayor influencia justo en las intersecciones, luego imponemos también 
\begin{equation*}
    \omega_k(c) = s, \text{ donde } c \in I \equiv \{p\in\R^3 : \phi(p) = \gamma(p)\},\ s\in \R.
\end{equation*}

El valor $s$ es el que deberemos ajustar para que $\smin$ cumpla nuestros requisitos. Fijado un $p\in B_{k}$, ya tenemos una primera aproximación para $\omega_k$ :
\begin{equation*}
    \omega(p) = s\left( 1-\frac{|\phi(p)-\gamma(p)|}{k} \right)^n = \begin{cases}
        s\left( 1-\frac{\phi(p)-\gamma(p)}{k}\right)^n,\ \phi(p)>\gamma(p), \\[10pt]
        s\left( 1+\frac{\phi(p)-\gamma(p)}{k}\right)^n,\ \phi(p)\le \gamma(p)\\[10pt]
    \end{cases}  ,\ s\in\R,\ n\in\N,
\end{equation*}
donde hemos añadido el parámetro $n$ para añadir más control sobre el resultado final.\newline 
\begin{figure}[!h]
     \begin{minipage}[c]{0.49\linewidth}
        \centering
        \includegraphics[width=0.95\textwidth]{Plantilla-TFG-master/img/smin_1.png}
        \caption{$k=0.6$}
     \end{minipage}
     \begin{minipage}[c]{0.49\linewidth}
        \centering
        \includegraphics[width=0.95\textwidth]{Plantilla-TFG-master/img/smin_2.png}
        \caption{$k=0.1$}
     \end{minipage}
     \caption{Primera aproximación de $smin(p)$ con $s=0.05$ y $n=2$}
     \label{fig:smooth1}
\end{figure}

Nuestro objetivo es que $\smin$ tenga un aspecto natural y varíe de forma suave. Comprobemos propiedades que debería cumplir $\smin$ para ser $\mathcal{C}^1$ en cada entorno de $B_k$. Que es continua es evidente:
\begin{equation*}
    \phi(p)=\gamma(p) \implies \frac{\phi(p)-\gamma(p)}{k} = 0\implies \omega_k(p) = s
\end{equation*}

Otra condición necesaria es que sus derivadas parciales sean continuas. Estas son de la forma
\begin{align*}
    \frac{\partial \smin}{\partial x_i}(p) &= \begin{cases}
        \frac{\partial \gamma}{\partial x_i}(p)+ sn\left(1-\frac{\phi(p)-\gamma(p)}{k}\right)^{n-1}\left(\frac{ \frac{\partial \phi}{\partial p}(p)-\frac{\partial \gamma}{\partial p}(p)}{k}\right),\ \phi(p)>\gamma(p), \\[10pt] 
        \frac{\partial \phi}{\partial x_i}(p)+ sn\left(1-\frac{\phi(p)-\gamma(p)}{k}\right)^{n-1}\left(\frac{ \frac{\partial \phi}{\partial p}(p)-\frac{\partial \gamma}{\partial p}(p)}{k}\right),\ \phi(p)\le\gamma(p)
    \end{cases},\ i=1,2,3.
\end{align*}

Por tanto, para que se cumpla la condición imponemos 
\begin{align*}
     \frac{\partial \phi}{\partial x_i} - sn\left(1+\frac{\phi-\gamma}{k}\right)^{n-1}\left(\frac{\frac{\partial \phi}{\partial x_i}-\frac{\partial \gamma}{\partial x_i}}{k}\right) &= \frac{\partial \gamma}{\partial x_i} + sn\left(1-\frac{\phi-\gamma}{k}\right)^{n-1}\left(\frac{\frac{\partial \phi}{\partial x_i}-\frac{\partial \gamma}{\partial x_i}}{k}\right)\\[10pt]
     \cancel{\frac{\partial \phi}{\partial x_i} - \frac{\partial \gamma}{\partial x_i}} &= 2sn\left(1-\frac{\phi-\gamma}{k}\right)^{n-1}\left(\frac{ \cancel{\frac{\partial \phi}{\partial x_i}-\frac{\partial \gamma}{\partial x_i}}}{k}\right)\\[10pt]
     s &= \frac{k}{2n}\left(1-\frac{\phi-\gamma}{k}\right)
\end{align*}

Evaluando en $c\in I$:
\begin{align*}
    s = \frac{k}{2n}\left(1-\frac{\cancelto{0}{\phi(c)-\gamma(c)}}{k}\right) \implies s = \frac{k}{2n}.
\end{align*}
    
Hemos llegado a la expresión final
\begin{align}
    \label{eq:correccion}
    \omega_k(p) &= \begin{cases}
        \frac{k}{2n}\left( 1-\frac{|\phi(p)-\gamma(p)|}{k} \right)^n,\ &|\phi(p)-\gamma(p)|\le k,\\[10pt]
        0,\ &\text{ otro caso }.
    \end{cases}\\[10pt] &= \frac{\Max\left( k - |\phi(p) - \gamma(p)|, 0\right)^n}{2n\cdot k^{n-1}}  ,\ s\in\R,\ n\in\N. 
\end{align}

Podemos observar los resultados en la \autoref{fig:smooth2}. Finalmente, para obtener una versión suavizada del máximo, es fácil comprobar que 
\begin{align*}
      \smax_{\phi,\gamma}\colon \R^3&\to \R,\\
      p &\mapsto -\smin_{-\phi,-\gamma}(p).
\end{align*}

Recogemos los resultados obtenidos a continuación.
\begin{figure}[!h]
     \begin{minipage}[c]{0.49\linewidth}
        \centering
        \includegraphics[width=0.95\textwidth]{Plantilla-TFG-master/img/smin_3.png}
        \caption{$k=0.1,\ n=2$}
     \end{minipage}
     \begin{minipage}[c]{0.49\linewidth}
        \centering
        \includegraphics[width=0.95\textwidth]{Plantilla-TFG-master/img/smin_4.png}
        \caption{$k=0.1,\ n=3$}
     \end{minipage}
     \caption{Resultado final de $smin(p)$ }
     \label{fig:smooth2}
\end{figure}





% Por tanto, dadas $\phi$ y $\gamma$, queremos obtener una versión suavizada de $\Min(\phi,\gamma)$ usando interpolación lineal, que llamaremos $\smin$ y tendrá la forma
% \begin{align*}
%           \smin\colon \R^3 &\to \R^3.\\
%           p &\mapsto h(p)\cdot \phi(p) + (1-h(p))\gamma(p) \text{, donde } h \colon \R^3 \to [0,1].
%     \end{align*}


% Pasamos a buscar $h$. Solo queremos modificar la función en los entornos de los puntos en los que intersecan $\phi$ y $\gamma$, de forma que para el resto de puntos debería ser $h=\{0,1\}$. Los puntos de intersección vienen dados como las soluciones de $m(p)=\gamma(p) - \phi(p)$. Podemos además acotar $m(p)$ en el intervalo $[0,1]$ usando $\Min$ y $\Max$, obteniendo un candidato a valor de $h(p)$:
% \begin{equation}
%     \Min\left(\Max\left(\phi(p)-\gamma(p),0\right),1\right) = \Min\left(\Max\left(m(p),0\right),1\right) \in [0,1]
% \end{equation}

% Sin embargo, podemos ver que la interpolación comienza justo en la intersección, mientras que nos gustaría que esto ocurriese antes. Modificamos la expresión anterior para hacer que la intersección sea el punto medio de la interpolación ($h=0.5$):
% \begin{equation}
%    \Min\left(\Max\left(m(p) + \frac{1}{2},0\right),1\right)
% \end{equation}

% Podemos ver los resultados de esta primera aproximación en la \autoref{fig:smooth1}.

% \begin{figure}[!h]
%      \begin{minipage}[c]{0.49\linewidth}
%         \centering
%         \includegraphics[width=0.95\textwidth]{Plantilla-TFG-master/img/smoothV1_a.png}
%         \caption{$h(p)=0$ en la intersección}
%      \end{minipage}
%      \begin{minipage}[c]{0.49\linewidth}
%         \centering
%         \includegraphics[width=0.95\textwidth]{Plantilla-TFG-master/img/smoothV1_b.png}
%         \caption{$h(p)=0.5$ en la intersección}
%      \end{minipage}
%      \caption{Primera aproximación de la obtención de $h(p)$}
%      \label{fig:smooth1}
% \end{figure}

% Observamos que ahora tenemos un nuevo problema

\begin{definicion}[Operaciones Booleanas Suavizadas]
    Sean $A$ y $B$ isosuperficies generadas por $\phi$ y $\gamma$ respectivamente. Definimos los SDF para las operaciones booleanas suavizadas como sigue.
    \begin{itemize}
        \item \textbf{Unión suavizada: } $\sdf_{unionS}(p) = \Min(\phi(p),\gamma(p)) - \frac{\Max\left( k - |\phi(p) - \gamma(p)|, 0\right)^n}{2n\cdot k^{n-1}}$,
        \item \textbf{Intersección suavizada: } $\sdf_{interS}(p) = -\Min(-\phi(p),-\gamma(p)) + \frac{\Max\left( k - |\phi(p) - \gamma(p)|, 0\right)^n}{2n\cdot k^{n-1}}$,
        \item \textbf{Diferencia suavizada: } $\sdf_{difS}(p) = -\Min(-\phi(p),\gamma(p)) + \frac{\Max\left( k - |\phi(p) + \gamma(p)|, 0\right)^n}{2n\cdot k^{n-1}}$,
    \end{itemize}

    donde $k\in \R^+_0$ controla la influencia del suavizado.        
\end{definicion}

Observamos que los operadores definidos en la \autoref{p:boolean} no son más que un caso particular de estos últimos cuando $k\to 0$. Este método se puede generalizar para obtneer\newline

% \begin{equation*}
%     \lim_{k\to 0} \frac{\Max\left( k - |\phi(p) - \gamma(p)|, 0\right)^n}{2n\cdot k^{n-1}} = \begin{cases}
%         \lim_{k\to 0 } \frac{0}{2n\cdot k^{n-1}} = 0,\ p\notin B_{p,k}\\[10pt]
%          \lim_{k\to 0 } \frac{\left( k - |\phi(p) - \gamma(p)|\right)^n}{2n\cdot k^{n-1}}
%     \end{cases} 
% \end{equation*}

Este método para obtener una versión suavizada de las funciones $\Min$ y $\Max$ no es la única. Hemos elegido esta debido a que su deducción es bastante natural, el efecto que tiene el valor $k$ sobre el resultado final es intuitivo y, sobretodo, porque es eficiente. En el artículo que hemos mencionado al inicio de la sección, Íñigo Quílez \cite{article:smooth} presenta otras tres alternativas a esta versión, a la cual él se refiere como \qq{mínimo suavizado polinomial}, y que también son compatibles con \textit{raymarching}. Si denotamos ahora $\smin(\phi(p), \gamma(p)) = \smin_{\phi,\gamma}$, estas versiones son:

\begin{itemize}
    \item \textbf{Mínimo suavizado exponencial:} $\smin(a,b) = \frac{-\log_2\left( 2^{-ka} + 2^{-kb} ) \right)}{k}$,
    \item \textbf{Mínimo suavizado potencial:} $\smin(a,b) = \left(\frac{a^k \cdot b^k}{a^k + b^k}\right)^{1/k}$,
    \item \textbf{Mínimo suavizado por raíz:} $\smin(a,b)= \frac{a +b - \sqrt{(a-b)^2+k}}{2}$.
\end{itemize}

La principal ventaja de la versión polinomial respecto a estas es que es la más rápida al ser sus cálculos computacionalmente más baratos. Por otro lado tanto la exponencial como la potencial permiten ser adaptadas fácilmente para calcular el mínimo de un conjunto arbitrario de puntos, útil cuando se trabaja con patrones de voronoi o nubes de puntos. Además, la versión exponencial produce siempre el mismo resultado independientemente del orden en el que se aplique. Es decir,
\begin{equation*}
    \smin(a,\smin(b,c)) = \smin(b,\smin(a,c)).
\end{equation*}

En la \autoref{fig:smoothVS} podemos ver un ejemplo de uso de estas versiones, en las que además se ha usado el valor de $w_k$ de la ecuación \autoref{eq:correccion} para interpolar la componente difusa de ambas primitivas usando el método \texttt{mix} de GLSL. Como vemos, no hay diferencias notables entre las distintas versiones, así que nos quedaremos con el método más eficiente: el polinómico.
\begin{figure}[htbp]
    \centering
    \begin{subfigure}[b]{0.25\textwidth}
        \centering
        \includegraphics[width=\textwidth]{Plantilla-TFG-master/img/unionMethodOG.png}
        \caption{Polinomial, $k=1.5,\ n=2$}
    \end{subfigure}
    \hfill
    \begin{subfigure}[b]{0.25\textwidth}
        \centering
        \includegraphics[width=\textwidth]{Plantilla-TFG-master/img/unionMethodExp.png}
        \caption{Exponencial, $k=2.5$}
    \end{subfigure}
    \hfill
    \begin{subfigure}[b]{0.25\textwidth}
        \centering
        \includegraphics[width=\textwidth]{Plantilla-TFG-master/img/unionMethodRoot.png}
        \caption{Raíz, $k=1$}
    \end{subfigure}
    
    \caption{Diferentes versiones de la unión suavizada}
    \label{fig:smoothVS}
\end{figure}

\subsection{Operaciones afines}
Pasamos ahora a estudiar otro tipo de operaciones que nos permitirán aplicar movimientos rígidos y cambios de escala a las primitivas en la escena. A diferencia de los operadores booleanos que eran binarios, estas operaciones se aplican a una única primitiva, y se basarán en aplicar una transformación $t:\R^3\to \R^3$ a cada punto de la isosuperficie $S_{\phi}$ para obtener la transformada $S_{\gamma}$. Si queremos saber si un punto $q\in\R^3$ está en $S_{\gamma}$, tenemos que comprobar si su preimagen por la transformación pertenece a $S_{\phi}$. Por tanto, bastará evaluar el SDF original en $t^{-1}(p)$:
\begin{equation*}
    \gamma(p) = \phi(t^{-1}(p)).
\end{equation*}

Esto funciona bien para transformaciones como las traslaciones o rotaciones, ya que son movimientos rígidos y mantienen las distancias. Sin embargo, este no es el caso del escalado, ya que si tomamos $l(p) = sp$ con $s\in \R^+_0$:
\begin{equation*}
    \Vert p-p'\Vert = d \implies  \Vert l(p)-l(p')\Vert = \Vert sp-sp'\Vert = s\Vert p-p'\Vert = s\cdot d,\  \text{ donde } p,p' \in S_{\phi}.
\end{equation*}

Como las distancias se escalan, deberemos hacer lo propio con el nuevo SDF, aplicándole el mismo factor de escalado $s$ como muestra la \autoref{d:afines}.

\begin{definicion}[Operaciones afines]\label{d:afines}
    Sea $A$ una isosuperficie. Definimos los SDF para las siguientes operaciones.
    \begin{itemize}
        \item \textbf{Traslación de vector $\boldsymbol{v}$: } $\sdf_{traslacion}(p) = \sdf_{A}(p - v)$,
        \item \textbf{Escalado uniforme de dimensiones $\boldsymbol{s}$: } $\sdf_{escalado}(p) = \sdf_{A}(p/s)\cdot s$,
        \item \textbf{Rotaciones de ángulo $\boldsymbol{\alpha\in \R}$ sobre los ejes $\boldsymbol{x,y,z}$: }
        \begin{align*}
            \sdf_{rotX}(p) &= \sdf_{A}(R_x^{-1}(\alpha)\cdot p^t),\ R_x(\alpha) = 
            \begin{pmatrix}
                1&0&0\\
                0&\cos(\alpha) & -\sin(\alpha) \\
                0&\sin(\alpha) & \cos(\alpha) 
                \end{pmatrix},\\[10pt] 
            \sdf_{rotY}(p) &= \sdf_{A}(R_y^{-1}(\alpha)\cdot p^t),\ R_y(\alpha) = \begin{pmatrix}
            \cos(\alpha) &0& \sin(\alpha)\\
            0&1&0\\
            -\sin(\alpha) &0& \cos(\alpha) 
            \end{pmatrix},\\[10pt]
            \sdf_{rotZ}(p) &= \sdf_{A}(R_z^{-1}(\alpha)\cdot p^t),\ R_z(\alpha) = \begin{pmatrix}
            \cos(\alpha) & -\sin(\alpha) & 0\\
            \sin(\alpha) & \cos(\alpha) & 0\\
            0&0&1
            \end{pmatrix}.
        \end{align*}
        
    \end{itemize}
\end{definicion}

\subsection{Operaciones deformantes}
Siguiendo el mismo razonamiento, podemos definir operaciones que modifiquen la geometría de la superficie aplicando rotaciones o traslaciones al punto en el que se evalúa la SDF original. De esta forma podemos obtener operadores que de otra forma sería mucho más complicado implementar, como la torsión o el redondeo de bordes.

\begin{definicion}[Operaciones Deformantes]
    Sea $A$ una isosuperficie. Definimos los SDF para las siguientes operaciones.
    \begin{itemize}
        
        \item \textbf{Torsión: } $\sdf_{torsion}(p) = \sdf_{A}(p')$, con $p' = R_z(ky)\cdot (x,z,y)^t$,
        \item \textbf{Plegado: } $\sdf_{plegado} =\sdf_{A}(p')$, con $p' = R_z(kx)\cdot p^t$,
        \item \textbf{Redondeo: } $\sdf_{redondeo}(p) = \sdf_{A}(p) - k$,
        \item \textbf{Desplazamiento: } $\sdf_{desplazamiento}(p) = \sdf_{A}(\delta(p))$,
        \item \textbf{Elongación de tamaño $\boldsymbol{h\in \R^3}$: } $\sdf_{elongacion}(p) = \sdf_{A}(p')$, con $p' = p - c(p, -h, h)$,
    \end{itemize}

    donde
    \begin{itemize}
        \item $k\in \R^+_0$ controla la intensidad de la deformación,
        \item $\delta\colon \R^3\to \R^3$ es un patrón de desplazamiento,
        \item $R_z(\alpha)\in \mathcal{M}_3(\R)$ es la matriz de rotación de ángulo $\alpha$ sobre el eje $z$ dada en la \autoref{d:afines},
        \item $c\colon \R^3\times \R^3 \times \R^3 \to \R^3,\ c(x,a,b)$ acota cada componente de $x$ entre las de $a$ y $b$.
    \end{itemize}
\end{definicion}

\subsection{Operaciones de repetición}
También podemos usar la técnica de cambiar el punto en el que evaluamos el SDF para, en lugar de modificar la geometría original, añadir copias de la primitiva identificando varios puntos con uno que pertenezca a la superficie. La manera más inmediata de conseguir esto es a través de la función valor absoluto, que nos permitirá identificar la componente de cada punto con su opuesta para generar simetrías, y el operador módulo, que identificará puntos a una distancia fija en cada eje.

\begin{definicion}[Operadores de Posicionamiento]\label{d:posicionamiento}
    Sea $A$ una isosuperficie. Definimos los SDF para las siguientes operaciones.
    \begin{itemize}
        \item \textbf{Simetrías sobre los ejes $\boldsymbol{x,y,z}$:}
        \begin{gather*}
            \sdf_{simX}(p) = \sdf_{A}(\vert x\vert, y, z),\quad \sdf_{simY}(p) = \sdf_{A}(x, \vert y\vert,  z),\\[5pt] \sdf_{simZ}(p) = \sdf_{A}(x,y,\vert z\vert),
        \end{gather*}
        \item \textbf{Simetrías sobre los planos $\boldsymbol{xy,xz,yz}$:}
        \begin{gather*}
            \sdf_{simXY}(p) = \sdf_{A}(\vert x\vert, \vert y\vert, z),\quad \sdf_{simXZ}(p) = \sdf_{A}(\vert x\vert, y,  \vert z\vert),\\[5pt]\sdf_{simYZ}(p) = \sdf_{A}(x,\vert y\vert ,\vert z\vert),
        \end{gather*}
        \item \textbf{Repetición $\boldsymbol{l\in \N^3}$ veces en los ejes $\boldsymbol{x,y,z}$ con separación $\boldsymbol{s\in\R}$:} 
        \begin{equation*}
            \sdf_{rep}(p) = \sdf_{A}(p - s\cdot c\left(r\left(\frac{p}{s}\right), -l, l\right),
        \end{equation*}
        \item \textbf{Repetición infinita:}
        \begin{equation*}
            \sdf_{repInf}(p) = \sdf_{A}\left((p+\frac{l}{2}\mod l )- \frac{l}{2}\right),
        \end{equation*}
    \end{itemize}
    donde
    \begin{itemize}
        \item $c\colon \R\times\R\times\R\to \R,\ c(x,a,b) = \Min(\Max(x, a), b)$ acota $x$ en $[a,b]$,
        \item $r\colon \R^3 \to \R^3$ redondea las componentes de un vector a sus enteros más cercanos.
    \end{itemize}
\end{definicion}

\begin{observacion}
    Hay casos en los que los SDF definidos en la \autoref{d:posicionamiento} podrían no ser exactos, al igual que ocurría con la intersección y la diferencia en la \autoref{p:boolean}:
    \begin{itemize}
        \item Para las simetrías, cuando el objeto interseca el plano de simetría,
        \item Para las repeticiones, cuando las dimensiones del objeto sean mayores o iguales a $l/2$.
    \end{itemize}
\end{observacion}

Este tipo de operaciones evidencia el potencial que tienen las SDFs en cuanto a eficiencia, ya que podemos visualizar miles de objetos al precio de uno. Por ejemplo, podríamos generar un campo de césped a partir de una única brizna de hierba.\newline



\section{Obtención a partir de ecuación implícita}
Empezábamos el capítulo diciendo que una de las representaciones más comunes de una superficie es a través de ecuaciones implícitas, pero hasta ahora nos hemos centrado en estudiar un subconjunto de esta familia. Si intentásemos aplicar el algoritmo de \textit{raymarching} a una función implícita cualquiera podríamos observar que el resultado presenta defectos, tales como deformaciones o grietas, o que incluso no se visualiza. Veamos qué podemos hacer para, dada una función $\phi$ cualquiera, obtener información aproximada de $S_\phi$ \cite{article:aprox}. Esto nos será útil cuando no conozcamos o no podamos calcular explícitamente la función distancia con signo de una superficie, pero sí su ecuación implícita.

\begin{proposicion}\label{p:aproxImp}
    Sea $\phi\colon \R^3\to\R$ una función infinitamente diferenciable. Entonces
    \begin{equation*}    
        \vert \sdf_{S_\phi}(p)\vert \ge \frac{\vert \phi(p)\vert}{\Vert \nabla\phi(p)\Vert}.
    \end{equation*}
\end{proposicion}
\begin{proof}
    Fijamos el punto $p$ del cual queremos aproximar la distancia a $S_{\phi}$. Sea $q$ el punto de $S_\phi$ más cercano a $p$ y $v=\vec{pq}$. Queremos calcular la distancia de $p$ a $S_\phi$, que será justamente $\Vert v\Vert$. Como $\phi$ es infinitamente diferenciable, podemos realizar el desarrollo de Taylor de $\phi$ centrado en $p$ y evaluado en $q=p+v$:
    \begin{equation*}
        \phi(p+v) = \phi(p) + \nabla\phi(p)\cdot (p+v -p) + \mathcal{O}(\vert p+v-p)\vert^2) = \phi(p) + \nabla\phi(p)\cdot v + \mathcal{O}(\vert v\vert^2).
    \end{equation*}
    Suponemos ahora que $p$ está cerca de $S_\phi$, de forma que existe un $\varepsilon>0$ tal que  $\Vert v\Vert < \varepsilon$, y podemos obviar el residuo. Como $\phi(q)=0$, tenemos que
    \begin{equation*}
        0 = \vert \phi(p+v)\vert \approx \vert \phi(p) + \nabla\phi(p)\cdot v \vert \ge \vert \phi(p)\vert - \vert \nabla\phi(p)\cdot v \vert \ge \vert \phi(p)\vert - \Vert \nabla\phi(p)\Vert\cdot \Vert v \Vert,
    \end{equation*}
    donde hemos usado la desigualdad triangular y la linealidad del producto escalar. De esta expresión, finalmente deducimos que
    \begin{equation*}
        \Vert v\Vert \ge \frac{\vert \phi(p)\vert}{\Vert \nabla\phi(p)\Vert}.\qedhere
    \end{equation*}
\end{proof}
% \begin{corolario}
%     Sea $\phi\colon \R^3\to \R$ lipschitziana con constante $L$. Entonces
%     \begin{equation*} 
%         \vert \sdf_{S_\phi}(p)\vert \ge \frac{\vert \phi(p)\vert}{ L}.
%     \end{equation*}
% \end{corolario}

Este resultado solo nos proporciona una cota inferior de la función distancia (sin signo). En nuestro caso esto es suficiente, pues esta nos sigue permitiendo representar la frontera de $S_{\phi}$, ya que proporciona una estimación conservadora de la distancia a ella. En su artículo \cite{art:impSdf}, Pierre-Alain Fayolle describe un método para obtener una función distancia con signo asociada a una superficie implícita que representa de manera exacta su frontera. Para ello, la descompone 
\begin{equation*}
    \sdf_{S_{\phi}}(p;\theta) = \phi(p)g(p;\theta)\quad \text{ o }\quad \sdf_{S_{\phi}}(p;\theta) = \sign(\phi(p))g(p;\theta),
\end{equation*}
donde $g$ es una función paramétrica de parámetros $\theta$ y $\sign$ es una versión suavizada de la función signo, por ejemplo $\sign(x) = \tanh{(kx)}$ con $k\in\R$. Para obtener la expresión de $g$ introduce la función $\phi$ en la capa final de una red neuronal entrenada para  minimizar una función pérdida asociada a la función distancia con signo, y para ajustar $\theta$ expresa $\sdf_{S_{\phi}}(p;\theta)$ como la solución de un problema variacional. No obstante, esta técnica está fuera del ámbito de este trabajo, de forma que nos limitaremos a usar la cota de la función distancia.

\section{Obtención a partir de ecuaciones paramétricas}

Ahora que sabemos representar las superficies generadas por una ecuación implícita cualquiera, nos proponemos ser capaces de representar también superficies definidas paramétricamente. Para ello fijaremos un anillo conmutativo $A$ y un conjunto de variables distintas $X=\{x_1,\dots, x_n\}$. Nuestro objetivo será, dado un conjunto $V\subseteq A^n$ por las ecuaciones paramétricas
\begin{align*}
    x_1 &= g_1(t_1,\dots, t_r),\\
    &\vdots \\
    x_n &= g_n(t_1,\dots, t_r),
\end{align*}
donde $g_i$ son polinomios de varias variables en $A$, obtener una ecuación implícita para $V$. En el caso que nos atañe $A=\R^3$, pero presentaremos todos los resultados de forma general.\newline

El contenido de esta sección está fuertemente basado en el libro Ideals, Varieties, and Algorithms de Cox, Little y O'Shea \cite{ideals_varieties}, el cual introduce de forma bastante completa resultados y algoritmos de álgebra conmutativa. Empezaremos explicando a qué nos referimos con polinomios de varias variables y recordando el concepto de ideal y sus propiedades. Después veremos que este problema equivale a uno de pertenencia a un ideal y cómo resolverlo usando la teoría de bases de Groebner.

\subsection{Polinomios en varias variables}
Estamos acostumbrados a trabajar con polinomios de una única variable como una suma o colección de monomios. Podemos mantener esta filosofía en el caso de varias variables adaptando el concepto que tenemos de estos.

\begin{definicion}
    Llamamos \textbf{monomio} en $X$ al producto de la forma
    $$x_1^{\alpha_1} \cdots x_n^{\alpha_n}\quad,\ \alpha_i \in \N,\ i\in\{1,\dots, n\}.$$
    Lo denotaremos como $X^{\alpha}$, y diremos que $\alpha\in \N^n$ es el \textbf{exponente} del monomio.
\end{definicion}

\begin{definicion}
    Definimos el \textbf{polinomio} $f:A^n\to A$ en $X$ con coeficientes en $A$ a toda combinación lineal finita de monomios
    \begin{align*}
        f = \sum_{\alpha\in \N^n} a_{\alpha} X^{\alpha}.
    \end{align*}

\end{definicion}

\begin{proposicion}
    El conjunto de polinomios es un anillo conmutativo. En concreto, para
     $$ f = \sum_{\alpha\in \N^n} a_{\alpha} X^{\alpha}\quad \text{ y }\quad g = \sum_{\beta\in \N^n} a_{\beta} X^{\beta},$$ 
     las operaciones internas del anillo son las siguientes.
    \begin{itemize}
        \item Suma heredada de $A$: $(f+g)(a) = f(a) + g(a)$.
        \item Producto de convolución: $(fg)(a) = \sum_{\alpha} \sum_{\beta+\gamma=\alpha} f(a)g(a)$.
    \end{itemize}
    Denotaremos como  $A[X] = A[x_1,\dots, x_n]$ a este anillo.
\end{proposicion}

En las siguientes secciones veremos que el problema de pertenencia de polinomios a un ideal se puede resolver mediante el procedimiento de la división. Este es bien conocido en polinomios de una variable, y ahora queremos extenderlo a un número arbitrario de ellas y varios divisores. Para ello en primer lugar necesitaremos una forma de ordenar los monomios que forman un polinomio. En una variable la forma \qq{natural} de comparar dos monomios es a través de su exponente. En el caso de varias variables la elección no es tan clara, y hay varias opciones que parecen igual de válidas. Vamos a formalizar el concepto de orden para introducir algunas de las posibilidades de las que disponemos. 

\begin{definicion}
    Un \textbf{orden total} sobre un conjunto $\Delta$ es una relación binaria $\le$ que cumple las siguientes propiedades.
    \begin{enumerate}
        \item Reflexiva: $a\le a,\ \text{para todo } a\in \Delta$.
        \item Transitiva: si $a\le b$ y $b\le c$ entonces $a\le c,\ \text{para todo } a,b,c\in \Delta$.
        \item Antisimétrica: si $a\le b$ y $b\le a$ entonces $a\le b,\ \text{para todo } a,b\in \Delta$.
        \item Completitud: $a\le b$ o $b\le a,\ \text{para todo } a,b\in \Delta$.
    \end{enumerate}
\end{definicion}
\begin{definicion}
    Un \textbf{orden admisible} es un orden total $\le$ sobre $\N^n$ cumpliendo
    \begin{enumerate}
        \item $(0,\dots ,0) \le \alpha,\ \text{para todo } \alpha \in \N^n$,
        \item si $\alpha < \beta$ entonces $\alpha + \gamma < \beta + \gamma,\ \text{para todo } \alpha,\beta,\gamma \in \N^n$.
    \end{enumerate}
\end{definicion}
\begin{proposicion}
    Todo orden admisible es un buen orden, esto es, todo subconjunto no vacío tiene un elemento mínimo.
\end{proposicion}

A partir de ahora siempre supondremos que todo orden que usemos es admisible en $\N^n$, luego podemos ordenar los monomios que conforman un polinomio ordenando sus exponentes según dicho orden. Veamos algunos de los órdenes más usados. 

\begin{definicion}
    Definimos el \textbf{orden lexicográfico} $\le_{\text{lex}}$ como
    \begin{equation*}
        \alpha \lex \beta \iff \begin{cases}
            \alpha  = \beta \\
            \quad\text{ó}   \\
            \alpha_i < \beta_i \text{, donde $i$ es el primer índice tal que } \alpha_i \neq \beta_i.
        \end{cases}
    \end{equation*}
\end{definicion}

% \begin{proof}
%     Es inmediato que es total, ya que $\le$ en $\N$ lo es. Veamos las otras condiciones.
%     \begin{enumerate}
%         \item Tomamos $\alpha \in \N^n$ tal que $\alpha\neq (0,\dots,0)$ y sea $i$ el primer índice tal que $\alpha_i \neq 0$. Entonces $0< \alpha_i$, de donde $(0,\dots, 0) \lex \alpha$.
%         \item Sean $\alpha$ y $\beta$ tales que $\alpha\lex \beta$, e $i$ el primer índice tal que $\alpha_i < \beta_i$
%     \end{enumerate}
% \end{proof}

\begin{definicion}
    Dado $\omega\in \N^n$, un orden admisible $\le$ se dice \textbf{$\boldsymbol{\omega}$-graduado} cuando
    \begin{equation*}
        \alpha\le \beta \text { implica que } \langle \alpha, \omega\rangle  < \langle \beta, \omega\rangle,
    \end{equation*}
    donde $\langle \alpha, \omega\rangle$ se llama el \textbf{$\boldsymbol{w}$-grado} de $\alpha$ y se define como
    \begin{equation}
        \langle \alpha, \omega\rangle = \alpha_1 \omega_1 + \cdots + \alpha_n \omega_n.
    \end{equation}
\end{definicion}

\begin{definicion}
    Dado un orden admisible $\le$, definimos el \textbf{orden $\boldsymbol{\omega}$-graduado asociado} como
    \begin{equation*}
        \alpha \le_{\omega} \beta \iff \begin{cases}
            \langle \alpha, \omega\rangle  < \langle \beta, \omega\rangle \\
            \quad\text{ó}   \\
           \langle \alpha, \omega\rangle = \langle \beta, \omega\rangle \text{ y } \alpha \le \beta.
        \end{cases}
    \end{equation*}
    Cuando $\omega = (1,\dots, 1)$ simplemente diremos que el orden es graduado, y usaremos las notaciones
    \begin{equation*}
        \le_{(1,\dots,1)} = \le_{\text{deg}},\quad (\lex)_{\text{deg}} = \le_{\text{deglex}},\quad (\lex)_{\omega} = \le_{\omega\text{-lex}}.
    \end{equation*}
\end{definicion}

\begin{definicion}
    Definimos el \textbf{orden lexicográfico graduado inverso} $\le_{\text{degrevlex}}$ como
    \begin{equation*}
        \alpha \le_{\text{degrevlex}} \beta \iff \begin{cases}
            |\alpha| < |\beta| \\
            \quad\text{ó}   \\
            |\alpha| = |\beta| \text{ y } \alpha_i > \beta_i \text{, donde $i$ es el último índice tal que } \alpha_i \neq \beta_i.
        \end{cases}
    \end{equation*}
\end{definicion}

\begin{proposicion}
    Sea $\le$ un orden admisible. Entonces $\le_{\omega}$ es admisible.
\end{proposicion}
\begin{proposicion}
    Los órdenes $\lex$ y $\le_{\text{degrevlex}}$ son admisibles.
\end{proposicion}

Una vez obtenida la noción de orden admisible, estamos en disposición de definir varios conceptos que nos resultarán imprescindibles para la manipulación de polinomios multivariable.

\begin{definicion}
    Sea $f= \sum_{\alpha} a_{\alpha} X^{\alpha}$ un polinomio y $\le$ un orden admisible. Definimos los siguientes conceptos asociados a $f$.
    \begin{itemize}
        \item \textbf{Exponente:} $exp(f) = \Max_{\le}(\alpha)$.
        \item \textbf{Monomio líder:}  $\lmf = X^{\expf}$.
        \item \textbf{Coeficiente líder:} $\lcf = a_{\expf}$.
        \item \textbf{Término líder:} $\ltf = \lcf \cdot \lmf$.
    \end{itemize}
\end{definicion}

\begin{definicion}
    Dado un monomio $X^{\alpha}$, definimos su \textbf{grado} como $\vert \alpha\vert = \alpha_1+\cdots + \alpha_n$. En el caso de un polinomio $f\in A[X]$, diremos que su grado es el grado de su monomio líder, y lo notaremos como $\deg(f)$.
\end{definicion}

Antes de presentar el algoritmo de la división, nos cercioramos de que esta operación siempre tiene sentido con el siguiente teorema.
\begin{teorema}[Algoritmo de división]
    Sea $F=\{f_1,\dots, f_s\} \subset A[X]$. Todo polinomio $f\in A[X]$ se puede expresar como
    \begin{equation*}
        f = q_1f_1 + \cdots + q_sf_s + r,
    \end{equation*}
    donde $q_i, r\in A[X]$ y $r=0$ ó $exp(r)\le exp(f)$. Llamaremos a $r$ el resto de dividir $f$ por $F$, y lo notaremos $\rem(f, [F]) = r$. Además, cuando $r=0$ diremos que $f$ reduce a $0$, y escribiremos $f \stackrel{F}{\to} 0$.
\end{teorema}

En otras palabras, podemos dividir $f$ entre cualquier conjunto de polinomios $F=\{f_1, \dots, f_s\}$ para expresarlo como combinación de sus elementos multiplicados por ciertos coeficientes polinómicos. El método es similar al usado en una variable, consistente en intentar reducir el monomio líder de $f$ restándole un múltiplo de cierto $f_i$. Para encontrar este $f_i$ simplemente se recorre el conjunto $F$ hasta encontrar uno válido, y de no haberlo se pasa el término líder al resto y se continua con el siguiente. Esto es justo lo que hace el \autoref{a:division}. Cabe destacar que esta forma de buscar el $f_i$ hace que la descomposición de $f$ obtenida no sea única, pues la elección dependerá de la posición que ocupen los divisores en el conjunto $F$, y por tanto del orden elegido. Más adelante veremos que la elección del orden influye en el resultado de más algoritmos.

\SetKwComment{Comment}{/* }{ */}
\begin{algorithm}[hbt!]
    \caption{División de polinomios en varias variables}\label{a:division}
    \KwData{dividendo $f$, divisores $F = \left[ f_1, \dots, f_s\right]$}
    \KwResult{Tupla con el resto $r$ y los coeficientes $q_i$ para cada $f_i\in F$}
    $p\gets f$
    
    $\left[q_1,\dots, q_s\right] \gets \left[0,\dots, 0\right]$
    
    $r\gets 0$

    \While{$p \neq 0$}{
        $\text{divisorEncontrado} \gets false$
        \For{$f_i \in F$} {
            \If{$\text{exp}(p) = \text{exp}(f_i) + \alpha$}{
                $q_i\gets q_i + \frac{\text{lc}(p)}{\text{lc}(f_i)} X^{\alpha}$
                
                $p \gets p - f_i \cdot \frac{\text{lc}(p)}{\text{lc}(f_i)} X^{\alpha}$
                
                $\text{divisorEncontrado} \gets true$
            }
        }
        \If{$!\text{divisorEncontrado}$}{
            $r \gets r + \text{lt}(p)$
            
            $p \gets p - \text{lt}(p)$
        }
    }
    \Return{$\left[r,q_1,\dots, q_s\right]$}
\end{algorithm}

\subsection{Bases de Groebner}
Ya tenemos claras las ideas sobre qué es un polinomio en varias variables, así que ahora pasamos a repasar el concepto de ideal y cómo podemos usar las bases de Groebner para trabajar con ellos en el caso de ideales de polinomios.
\begin{definicion}
    Decimos que $\emptyset \neq I \subseteq A[X]$ es un \textbf{ideal} de $A[X]$ si
    \begin{enumerate}
        \item $0\in I$,
        \item $a+b\in I,\ \text{para todo } a,b\in I$,
        \item $af\in I,\ \text{para todo } a\in I,\ \text{para todo } f\in A[x]$.
    \end{enumerate}
    En ese caso escribiremos $I\le A$.
\end{definicion}

\begin{proposicion}
    Dados los ideales $I,J\le A[X]$, son también ideales de $A[X]$:
    \begin{enumerate}
        \item $I+J = \{f+g : f\in I, g\in J\}$,
        \item $IJ = \{f_1g_1 + \cdots + f_tg_t : f_i\in I, g_i\in J, 1\le i \le t\}$,
        \item $I\cap J = \{h: h\in I \text{ y } h\in J\}$.
    \end{enumerate}
\end{proposicion}

Podemos calcular estos ideales usando sus conjuntos de generadores.
\begin{definicion}
    Dado $F=\{f_1,\dots, f_s\}\subseteq A[X]$, el \textbf{ideal generado} por $F$ es
    \begin{equation*}
        \langle F \rangle = \{a_1f_1 + \cdots + a_sf_s : a_1,\dots, a_s\in A,\ f_1,\dots, f_s\in F\}\le A[X].
    \end{equation*}
    Diremos que $F$ es un \textbf{conjunto de generadores} de $I$.
\end{definicion}

\begin{proposicion}
    Sean $I=\langle F\rangle$ y $J=\langle G\rangle$ ideales de $A[X]$. Entonces
    \begin{enumerate}
        \item $I+J = \langle F\cup G\rangle$,
        \item $IJ = \langle fg : f\in F, g\in G \rangle$,
        \item $I\cap J = \langle tF, (1-t)G \rangle \cap A[x_1,\dots, x_n]$.
    \end{enumerate}
\end{proposicion}

Pasamos a presentar el concepto de base de Groebner asociada a un ideal. Podemos pensar que una base de Groebner es a un ideal lo que un sistema de generadores a un espacio vectorial: un subconjunto a partir del cual podemos obtener el total.
\begin{definicion}
    Sea $I\le A[X]$. Denotamos el conjunto de los términos líder de $I$ como
    \begin{equation*}
        \lt(I) = \{\lt(f) : f\in I\}.
    \end{equation*}
\end{definicion}

\begin{definicion}
    Dado $I\le A[X]$, diremos que $G = \{g_1,\dots, g_s\}\subseteq I$ es una \textbf{base de Groebner} para $I$ si 
    $$\langle \lt(I)\rangle = \langle \lt(g_1),\dots, \lt(g_t) \rangle.$$
\end{definicion}

La analogía con el sistema de generadores de un espacio vectorial nos conduce de forma natural a la pregunta de si habrá también un análogo al concepto de base, y si dado un ideal $I$ siempre existirá una base de Groebner asociada a este. La respuesta es afirmativa en ambos casos.
\begin{definicion}
    Dada $f\in A[X]$, definimos su \textbf{soporte} como $$\supp(f) = \{ \alpha\in\Nn : f(\alpha) \neq 0\}.$$
\end{definicion}

\begin{definicion}
    Sea $I\le A[X]$. Diremos que $G$ es una \textbf{base de Groebner reducida} para $I$ si para todo $g\in G$ se cumple
    \begin{enumerate}
        \item $\lcg=1$,
        \item $\supp(g) \cap \left(\exp(G\setminus\{g\}) + \Nn \right) = \emptyset$.
    \end{enumerate}
    Es decir, una base será reducida si ningún elemento se puede expresar como combinación del resto. Esto equivale a que
    \begin{equation*}
        \rem\left(g, G\setminus \{g\}\right) \neq 0,\ \text{para todo } g\in G.
    \end{equation*}
\end{definicion}

\begin{definicion}
    Sea $M\le \Nn$. Decimos que $A$ es un \textbf{conjunto generador minimal} de $M$ si
    \begin{equation*}
        M = A + \Nn \quad \text{ y } \quad M\neq (A\setminus \{a\}) + \Nn,\ \text{para todo } a \in A.
    \end{equation*}
\end{definicion}

\begin{lema}\label{l:minimal}
    Todo ideal tiene un único conjunto generador minimal.
\end{lema}
\begin{lema}
    Sea $I\le A$. Se cumple que $a-b\in I,\ \text{para todo } a,b\in I$. 
\end{lema}
\begin{proof}\label{l:resta}
    Basta observar que tomando $-1 \in A$ obtenemos que $b\cdot (-1) \in I$, de donde
    \begin{equation*}
        a-b = a+ b(-1) \in I.\qedhere
    \end{equation*}
\end{proof}
\begin{teorema}\label{t:reduce}
    Todo ideal $I$ admite una única base de Groebner reducida para un orden admisible dado.
\end{teorema}
\begin{proof}
    \mybox{Existencia} Sea $G$ un conjunto generador minimal de $\exp(I)$, que sabemos que existe por el \autoref{l:minimal}. Sea $g\in G$ y $r = \rem\left(g, [G\setminus \{g\}]\right)$. Tenemos que
    \begin{equation*}
        \exp(g) \notin \exp(G\setminus\{g\}) + \Nn \text{ luego } \exp(g)=\exp(r),
    \end{equation*}
    de donde
    \begin{equation*}
        \exp(G) = \exp\Big( (G\setminus \{g\})\cup \{r\}\Big).
    \end{equation*}
    Además $g-r\in \langle G\setminus \{g\} \rangle \subseteq I$, de forma que $r\in I$ y $G' = (G\setminus \{g\} \cup \{r\}$ es una base de Groebner de $I$ cumpliendo $\supp(r) \cap \Big(\exp(G'\setminus\{r\}) + \Nn \Big) = \emptyset$.
    Aplicando este procedimiento a cada elemento de $G$ obtenemos una base reducida de $I$.\\[5pt]
    \mybox{Unicidad} Sean $G_1,G_2$ dos bases minimales de $I$. Por el \autoref{l:minimal}, como $\exp(G_1) = \exp(G_2)$, dado cualquier $g_1\in G_1$, no existe ningún $g_2\in G_2$ tal que $\exp(g_1) = \exp(g_2)$. Por otro lado, se cumple
    \begin{enumerate}
        \item $ \supp(g_1-g_2) \subseteq \Big( \supp(g_1)\cup \supp(g_2) \Big) \setminus \{\exp(g_1)\}$,
        \item $\supp(g_i)\setminus \{\exp(g_i)\}\cap \Big(\{\exp(G_i) + \Nn\}\Big) = \varnothing,\ i\in\{1,2\}$,
    \end{enumerate}
    de donde
    \begin{equation*}
        \supp(g_1-g_2) \cap \left(\exp(G_1)+\Nn\right) = \varnothing.
    \end{equation*}
    Concluimos entonces que $\rem(g_1-g_2, G_1) = g_1-g_2$, y como por el  \autoref{l:resta} sabemos que $g_2-g_1\in I$, dicho resto será igual a cero, de forma que $g_1=g_2$ y $G_1 = G_2$.
\end{proof}

Terminamos la sección obteniendo un algoritmo para calcular la base de Groebner reducida para un ideal dado un conjunto de generadores suyo $G$. Este se basará en eliminar de $G$ aquellos polinomios cuyos exponentes podamos poner como combinación lineal del resto. Claro está que no podemos realizar esta comprobación directamente, y deberemos buscar alguna condición equivalente que sí podamos calcular.

\begin{definicion}
    Dados $\alpha,\beta \in \Nn$, definimos los términos
    \begin{itemize}
        \item \textbf{Mínimo común múltiplo}: $\lcm(\alpha,\beta) = \{\Max(\alpha_1, \beta_1),\dots, \Max(\alpha_n, \beta_n)\}$,
        \item \textbf{Máximo común divisor}: $\gcd(\alpha,\beta) = \{\Min(\alpha_1, \beta_1),\dots, \Min(\alpha_n, \beta_n)\}$.
    \end{itemize}
\end{definicion}

\begin{definicion}
    Sean $f,g \in A[X]$. Tomando $\alpha=\exp(f),\ \beta=\exp(g)$ y $\gamma = \lcm(\alpha,\beta)$, se define el \textbf{S-polinomio} de $f$ y $g$ como
    \begin{equation*}
        S(f,g) = \lc(g)X^{\gamma-\alpha}f - \lc(f)X^{\gamma-\beta}g.
    \end{equation*}
\end{definicion}

\begin{teorema}[Primer Criterio de Buchberger]\label{t:criterio}
    Sean $I\le A[X]$ y $G=\{g_1,\dots, g_t\}$ un conjunto de generadores de $I$. Entonces:
    \begin{equation*}
        G \text{ es base de Groebner para } I \iff \rem(S(g_i,g_j), G)=0,\ \text{para todo } 1\le i<j\le t.
    \end{equation*}
\end{teorema}

El algoritmo que usaremos para el cálculo de la base de Groebner se basará en este criterio. Sin embargo, antes de presentarlo estudiamos dos criterios adicionales \cite{criterio1,criterio2} que lo harán más eficiente descartando S-polinomios antes de comprobar su resto, ahorrando el cómputo de numerosas divisiones.

\begin{definicion}
    Sea $f= \sum_{\alpha} a_{\alpha} X^{\alpha}$ un polinomio cuyo monomio líder es $X^{\alpha^{(k)}}$ y $\le$ un orden admisible. Definimos el \textbf{segundo monomio líder} de $f$ como el monomio $X^{\alpha^{(i)}}$ de $f$ tal que
    \begin{equation*}
        X^{\alpha^{(i)}} \ge X^{\alpha^{(j)}},\ \text{ para todo } j\notin\{i,k\}. 
    \end{equation*}
    Lo denotaremos como $\sm(f)$.
\end{definicion}
\begin{teorema}[Criterios de Buchberger]\label{t:criterios}
    Sean $I\le A[X]$, $G\subseteq A[X]$ un conjunto de generadores de $I$, y $g_1,g_2 \in G$. Si se cumple cualquiera de las siguientes condiciones entonces  $S(g_1,g_2)\reduces 0$.
    \begin{enumerate}
        \item $\lcm(g_1,g_2) = \lm(g_1)\lm(g_2)$,
        \item existe un $f\in G$ tal que $\lm(f)\ \vert\ \lcm(g_1,g_2)$ y además
        \begin{enumerate}
            \item algún $S(g_i,f)\reduces 0\quad$ ó
            \item $\lm(f)\vert \frac{\lm(g_i)}{\gcd(g_1,g_2}$ y $\sm(g_j)\lm(f) \neq \sm(f)\lm(g_j)$,
        \end{enumerate}
        donde $i,j\in\{1,2\}$ e $i\neq j$.
    \end{enumerate}
    
\end{teorema}

Usando los criterios obtenidos obtenemos el \autoref{a:buchberger} para calcular la base de Groebner de cualquier ideal. La salida de este no es una base minimal, pero la demostración del \autoref{t:reduce} nos proporciona un método para reducir una base cualquiera a la minimal asociada. En el \autoref{a:minim} mostramos este procedimiento.\newline

\SetKwComment{Comment}{/* }{ */}
\begin{algorithm}[hbt!]
    \caption{Algoritmo de Buchberger optimizado}\label{a:buchberger}
    \KwData{polinomio $f$, conjunto de generadores $F = \left[ f_1, \dots, f_S\right]$}
    \KwResult{base de Groebner $G$}

    $G\gets F$\;

    \Repeat{$G' = G$}{
        $G'\gets G$\;
        \For{each pair $\{f,g\} \subseteq G'$} {
            \If{$\text{!Criterio 1}(f,g) \textbf{ AND } !\text{Criterio 2}(f,g, G')$}{
                $r\gets \rem(S(f,g), G')$\;
                \If{$r\neq 0$}{
                    $G\gets G\cup \{r\}$\;
                }
            }
        }
    }

    \Return{$G$}
\end{algorithm}

\begin{algorithm}[hbt!]
    \caption{Minimización de base de Groebner}\label{a:minim}
    \KwData{$G$ base a minimizar}

    $G\gets F$\;

    \ForEach {$g \in G$}{
        $g\gets g/\lc(g)$\;
        $r\gets \rem(g, [G\setminus \{g\}])$\;

        \If{$r\neq 0$}{
            $g \gets r$\;
        }
    }
\end{algorithm}

Ya somos capaces de obtener una base de Groebner minimal de cualquier ideal dado un conjunto de generadores suyo, pero en este proceso se toma una decisión que aún no hemos discutido: cómo se eligen las parejas $\{f,g\}$. Uno de los métodos más usados es la conocida como \textbf{estrategia normal}, debido a su simpleza y haber probado ser de las que completan más rápido el algoritmo, y consiste en tomar el par $f,g$ cuyo $\lcm(f,g)$ sea del menor grado posible según el orden admisible usado. Vemos por tanto que de nuevo la elección de un orden u otro nos proporcionará resultados diferentes, y en este caso esto se traduce en que la base reducida tenga muchos menos elementos en un orden que en otro.

\subsection{Teorema de implicitación}
Empezábamos la sección diciendo que el problema de implicitación equivalía al de pertenencia a un ideal. Antes de ver de qué ideal se trata tenemos que introducir unos últimos conceptos que nos ayuden a entender por qué.

\begin{definicion}Dado $F=\{f_1,\dots, f_s\} \subseteq A[X]$, llamamos \textbf{variedad afín} definida por $F$ al conjunto:
    \begin{equation*}
        \mathbb{V}(F) = \{(a_1,\dots, a_n)\in A^n : f_i(a_1,\dots, a_n)=0,\ \text{para todo } i\in\{1,\dots, s\}\}.
    \end{equation*}
\end{definicion}

\begin{proposicion}
    Sean $\mathbb{V}(F)$ y $\mathbb{V}(G)$ variedades afines. Entonces
    \begin{itemize}
        \item  $\mathbb{V}(F) =  \mathbb{V}(\langle F\rangle)$,
        \item $\mathbb{V}(F\cup G) = \mathbb{V}(F) \cap \mathbb{V}(G)$,
        \item $\mathbb{V}(FG) = \mathbb{V}(F) \cup \mathbb{V}(G)$.
    \end{itemize}
\end{proposicion}

\begin{proposicion}
    Sean los ideales $I,J\le A[X]$. Entonces  $\mathbb{V}(I \cap J) = \mathbb{V}(I) \cup \mathbb{V}(J)$.
\end{proposicion}
\begin{definicion}
    Sea $B\subseteq A^n$. Definimos el \textbf{ideal asociado} a $B$ como
    \begin{equation*}
        \mathbb{I}(B) = \{f\in A[X] : f(b_1,\dots, b_n) = 0,\ \text{para todo } (b_1,\dots, b_n)\in B\}.
    \end{equation*}
\end{definicion}

Para resolver el problema de implicitación deberemos aprender antes a eliminar variables de un ideal.
\begin{definicion}
    Dado $I\le A[x_1,\dots,x_n]$, definimos su \textbf{ideal de $l$-eliminación} como
    \begin{equation*}
        I_l = I \cap A[x_{l+1}, \dots, x_n] \le A[x_{l+1}, \dots, x_n].
    \end{equation*}
\end{definicion}

\begin{definicion}
    Decimos que un orden admisible $\le$ es un \textbf{orden de $l$-eliminación} si 
    $$\beta\le \alpha \text{ implica } \beta \in \N_l^n,\ \text{para todo } \alpha \in \N_l^n \text{ y } \text{para todo } \beta \in \Nn,$$
    donde $\N_l^n = \{\alpha\in \Nn \colon \alpha_i =0,\ 1\le i \le l\}$.
\end{definicion}

\begin{teorema}[Eliminación]
    Sea $I\le A[x_1,\dots,x_n]$ y $G$ una base de Groebner suya respecto a un orden $\le$ de $l$-eliminación. Entonces, una base de Groebner para $I_l$ viene dada por
    \begin{equation*}
        G_l = G\cap A[x_{l+1},\dots, x_n].
    \end{equation*}
\end{teorema}

% Ahora sí, veamos de qué ideal se trata.

Con estos resultados podemos decir que el problema de implicitación consiste en encontrar la variedad asociada a las ecuaciones paramétricas
\begin{equation*}
    \begin{cases}
    x_1 &= g_1(t_1,\dots, t_r),\\
    &\vdots \label{eq:paramEq} \\
    x_n &= g_n(t_1,\dots, t_r).
    \end{cases}
\end{equation*}
Si escribimos $g_i = f_i/q_i$ con $f_i,q_i \in A[t_1,\dots, t_r]$ para $i=1,\dots, r$, podemos definir la aplicación
\begin{align*}
        \phi \colon A^r\setminus W  & \to A^n,\\
        (a_1,\dots, a_r) & \mapsto \left( \frac{f_1(a_1,\dots, a_r)}{q_1(a_1,\dots, a_r)}, \dots, \frac{f_n(a_1,\dots, a_r)}{q_n(a_1,\dots, a_r)}\right),
    \end{align*}
donde $W=\mathbb{V}(q_1\cdots q_r)$. Veamos cómo encontrar la menor variedad que contiene la imagen de $\phi$ en el caso de que $q_i = 1$ para cada $i\in \{1,\dots, r\}$. 
\begin{teorema}[Implicitación Polinomial]\label{t:implicit}
    Dados $f_1,\dots, f_n \in A[t_1, \dots, t_r]$ con $A$ cuerpo infinito, sea
    \begin{align*}
        \phi \colon A^r  & \to A^n,\\
        (a_1,\dots, a_r) & \mapsto \left( f_1(a_1,\dots, a_r), \dots, f_n(a_1,\dots, a_r) \right).
    \end{align*}
    Definimos los ideales:
    \begin{itemize}
        \item $I = \langle x_1-f_1,\dots,  x_n-f_n\rangle \le A[t_1,\dots, t_r,x_1\dots, x_n]$,
        \item $J = I\cap A[x_1,\dots, x_n]$ el ideal de $r$-eliminación de $I$.
    \end{itemize}
    Entonces, $\mathbb{V}(J)$ es la menor variedad que contiene a $\phi(A^r)$.
\end{teorema}


La extensión al caso racional es la siguiente.
\begin{teorema}[Implicitación Racional]\label{t:implicitRac}
    Sea $f_1,\dots, f_n, q_1,\dots, q_n \in A[t_1, \dots, t_r]$ con $A$ cuerpo infinito, $W=\mathbb{V}(q_1,\dots, q_n)$ y
    \begin{align*}
        \phi \colon A^r\setminus W  & \to A^n,\\
        (a_1,\dots, a_r) & \mapsto \left( \frac{f_1(a_1,\dots, a_r)}{q_1(a_1,\dots, a_r)}, \dots, \frac{f_n(a_1,\dots, a_r)}{q_n(a_1,\dots, a_r)}\right).
    \end{align*}
     Definimos los ideales:
    \begin{itemize}
        \item $I = \langle q_1x_1-f_1,\dots,  q_nx_n-f_n, 1-q_1\cdots q_ny\rangle \le A[y,t_1,\dots, t_r,x_1\dots, x_n]$,
        \item $J = I\cap A[x_1,\dots, x_n]$ el ideal de $1+r$-eliminación de $I$.
    \end{itemize}
    Entonces, $\mathbb{V}(J)$ es la menor variedad que contiene a $\phi(A^r\setminus W)$.
\end{teorema}

\begin{observacion}
    En el caso $r=1$ y cuando $f_i$ y $q_i$ sean primos relativos para cada $1\le i \le n$, basta tomar
    $$I = \langle q_1x_1-f_1,\dots,  q_nx_n-f_n\rangle.$$
\end{observacion}

Con este resultado, una vez obtenida la variedad, si resulta que esta tiene un único generador este será una potencia la ecuación implícita, luego el ideal al que llevamos haciendo referencia desde el principio de la sección y del que queríamos comprobar la pertenencia es el ideal $J$ de los teoremas anteriores. El hecho de que no obtengamos la potencia exacta de la ecuación implícita no es problema, pues nos basta conocer donde se anula para poder representar la frontera de la superficie que genera.  Sin embargo, no tenemos asegurado que vaya a haber un único generador del ideal, de forma que la superficie satisfaría varias ecuaciones implícitas y no podría ser representada por una sola. A continuación presentamos un resultado que aporta información al respecto.

\begin{definicion}
    Dado un ideal $I\le A[X]$, definimos su radical como
    \begin{equation*}
        \sqrt{I} = \{f\in A[X] : f^m\in I \text{ para algún } m\in \N\}.
    \end{equation*}
\end{definicion}
\begin{proposicion}
    Sea $I\le A[X]$. Entonces $\sqrt{I}$ es un ideal y contiene a $I$.
\end{proposicion}
\begin{definicion}
    Decimos que un ideal $I\le A[X]$ es \textbf{radical} si $\sqrt{I} = I$.
\end{definicion}

\begin{proposicion}
    Sea $B\subseteq A^n$. Entonces $\mathbb{I}(B)$ es un ideal radical.
\end{proposicion}
\begin{teorema}[Nullstellensatz fuerte]
    Si $A$ es algebraicamente cerrado, dado un ideal $I\le A[X]$ se cumple
    \begin{equation*}
        \sqrt{I} = \mathbb{I}(\mathbb{V}(I)).
    \end{equation*}
\end{teorema}
\begin{proposicion}
    Sea $I\le A[X]$ y $f\in A[X]$. Entonces
    \begin{equation*}
        f\in \sqrt{I} \text{ si y solo si } \langle I \rangle + \langle 1-fy \rangle = A[X].
    \end{equation*}
\end{proposicion}
Así, si pudiéramos calcular el radical del ideal $J$ de los teoremas de implicitación y este tuviera un solo elemento, tendríamos asegurado que la variedad está generada por esa única ecuación implícita. Además, calculando el radical podríamos obtener también la potencia exacta de la ecuación implícita que obtuvimos con el \autoref{t:implicitRac}. Sin embargo, el cálculo del radical o su número de elementos es en general muy complicado, y no es una opción viable. Por tanto, lo que haremos en la práctica será simplemente aplicar el algoritmo y comprobar si efectivamente se obtiene un único generador, en cuyo caso contrario concluiremos que no podemos realizar la implicitación.\newline

Hay otros métodos diferentes que permiten abordar el problema de eliminación de variables, y por tanto el de implicitación. Uno especialmente interesante es el uso de resultantes, pues en casos específicos puede simplificar mucho la obtención de la ecuación implícita. En la siguiente sección estudiaremos como podemos usar de forma básica el resultante del sistema de ecuaciones \eqref{eq:paramEq} para obtener su representación implícita.




