% !TeX root = ../libro.tex
% !TeX encoding = utf8

\section{Preliminares}

\begin{definicion}
    Sea $\Omega \subseteq \R^3$ un espacio métrico. Una \textbf{función distancia} $d(x)$ es aquella que a cada punto de $\R^3$ le asigna la menor distancia a la frontera de $\Omega$:
    \begin{equation*}
        d(x) = \min(|x-x_i|),\ \forall x_i \in \delta\Omega
    \end{equation*}
\end{definicion}

\begin{definicion}[SDF]
    Una \textbf{función distancia con signo} es una función implícita $\phi(x)$ de forma que:
    \begin{equation*}
        \phi(x) = \begin{cases}
            d(x)  & , x\in \R^3\setminus \mathring{\Omega} \\
            -d(x) & , x\in \mathring{\Omega}
        \end{cases}
    \end{equation*}

    De forma general nos referiremos a esta función por sus siglas en inglés SDF (\textit{Signed Distance Field}).
\end{definicion}

\section{Raymarching}

\section{Iluminación}
Utilizaremos el modelo de reflexión de Phong, en el cual se trabaja con los siguientes elementos:
\begin{itemize}
    \item Luz ambiente $i_a$.
    \item Materiales definidos por las constantes $(k_s, k_d, k_a, \alpha)$, que representan los factores de reflexión especular, difusa y ambiente y el coeficiente de brillo respectivamente.
    \item Lista de fuentes de luz $(l_1, \dots, l_m, \dots, l_n)$
    \item Los vectores normalizados definidos para cada punto de la superficie $p$:
          \begin{itemize}
              \item $L_m$: vector director desde $p$ a cada fuente de luz $l_m$.
              \item $N$ vector normal a la superficie en $p$.
              \item $R_m$ dirección del rayo de luz reflejado especularmente desde la fuente $l_m$ en el punto $p$.
              \item $V$: dirección de $p$ a la posición del observador.
          \end{itemize}
\end{itemize}

Con estas variables, el color en $p$ vendrá dado por la fórmula:
\begin{equation*}
    I_p = k_a i_a + \sum_{m=0}^{n} \left(k_d\left(L_m\cdot N\right) + k_s\left(R_m\cdot V\right)^{\alpha} \right)
\end{equation*}

De los vectores anteriores, $L_m$ y $V$ se calculan trivialmente, y $R_m$ se puede obtener como
\begin{equation*}
    R_m = 2(L_m\cdot N)N - L_m
\end{equation*}

Sin embargo, dado que la superficie viene dada por un SDF, no podemos calcular $N$ de forma analítica, y deberemos buscar un método que únicamente haga uso de evaluaciones de $\phi$.

\begin{definicion}
    El \textbf{gradiente} de una función implícita $\phi$ es $\nabla\phi = \left(\frac{\partial \phi}{\partial x}, \frac{\partial \phi}{\partial y}, \frac{\partial \phi}{\partial z}\right)$
\end{definicion}

\begin{definicion}
    Dada una función $\phi(x,y,z)\colon \R^3 \to \R^3$, llamamos \textbf{isosuperfice} a la superficie $S_\phi$ dada por $\phi(x,y,z)=k,\ k\in \R$. Tomaremos $k=0$.
\end{definicion}


\begin{proposicion}\label{p:gradient_perp}
    $\nabla\phi$ es perpendicular a $S_\phi$.
\end{proposicion}

\begin{proof}
    Sea $s\in S_\phi$ arbitrario. Tomamos una curva parametrizada:
    \begin{align*}
        \alpha \colon [0,1] & \to S_\phi                             \\
        t                   & \mapsto \left(x(t), y(t), z(t) \right)
    \end{align*}

    cumpliendo $\alpha(0)=s$. Veamos que $\nabla\phi \perp s$:

    \begin{align*}
        \alpha(t)\subset S_\phi & \implies \phi(\alpha(t))=0                                                                                                                                                      \\
                                & \implies \nabla\phi(\alpha(t)) = \frac{\partial{F}}{\partial{x}}\frac{dx}{dt} + \frac{\partial{F}}{\partial{y}}\frac{dy}{dt} + \frac{\partial{F}}{\partial{z}}\frac{dz}{dt} = 0 \\
                                & \implies \langle \nabla\phi(x,y,z), \alpha'(t)\rangle = 0
    \end{align*}

    Evaluando en $t=0$ obtenemos el resultado.
\end{proof}

Para obtener el vector normal en un punto $p$, tomaremos un conjunto de puntos cercanos a dicho punto y haremos un promedio de la derivada en dichos puntos. La elección de estos puntos puede hacerse según una gran variedad de criterios. En nuestro caso, usaremos el \textbf{método del tetraedro}.

\begin{definicion}
    Dada $f:\R^3\to \R$ diferenciable, $p = (x,y,z)$, $v\in \R^3$, definimos la \textbf{derivada direccional} en $p$ con dirección $v$ a:
    \begin{equation*}
        \nabla_v f(x) = \nabla f(x) \cdot v = \frac{\partial{f}}{\partial{x}}v_x + \frac{\partial{f}}{\partial{y}}v_y + \frac{\partial{f}}{\partial{z}}v_z
    \end{equation*}
\end{definicion}

Para el cálculo de cada parcial de $f$ podemos utilizar la definición de derivada. Por ejemplo, para la primera componente:
\begin{equation*}
    \frac{\partial{f}}{\partial{x}}(p) v_x = \lim_{h\to 0}\frac{f(p + (h,0,0)) - f(p)}{h} v_x
\end{equation*}

\begin{proposicion}
    Dado $p\in S_\phi$ su vector normal $N$ puede obtenerse como
    \begin{equation*}
        N = \sum_{i=0}^3 k_i\cdot f(p + hk_i)
    \end{equation*}
    donde $k_0 = (1,-1,-1), k_1 = (-1,-1,1), k_2=(-1,1,-1), k_3=(1,1,1)$.
\end{proposicion}

\begin{proof}
    Por la proposicion , basta comprobar que el vector $N$ definido es $\nabla \phi(p)$.

    \begin{align*}
        N & = \sum_{i=0}^3 k_i\cdot f(p + hk_i) = \sum_{i=0}^3\left[ k_i\cdot f(p + hk_i) + k_i\cdot f(p) \right] \\
          & = \sum_{i=0}^3 \left[k_i\left( f(p+hk_i - f(p)) \right) \right] = \sum_{i=0}^3(f_i \nabla_{k_i}f(x))
    \end{align*}
\end{proof}

\section{Operadores}
\subsection{Transformaciones afines}
\subsection{Booleanos}
\subsection{Fusiones}
\subsection{Deformaciones}