% !TeX root = ../libro.tex
% !TeX encoding = utf8

\section{Preliminares}

\begin{definicion}
    Sea $\Omega \subseteq \R^3$ un espacio métrico. Una \textbf{función distancia} $d(x)$ es aquella que a cada punto de $\R^3$ le asigna la menor distancia a la frontera de $\Omega$:
    \begin{equation*}
        d(x) = \min(|x-x_i|),\ \forall x_i \in \delta\Omega
    \end{equation*}
\end{definicion}

\begin{definicion}[SDF]
    Una \textbf{función distancia con signo} es una función implícita $\phi(x)$ de forma que:
    \begin{equation*}
        \phi(x) = \begin{cases}
            d(x)  & , x\in \R^3\setminus \mathring{\Omega} \\
            -d(x) & , x\in \mathring{\Omega}
        \end{cases}
    \end{equation*}

    De forma general nos referiremos a esta función por sus siglas en inglés SDF (\textit{Signed Distance Field}).
\end{definicion}

\section{Raymarching}

\section{Iluminación}
Utilizaremos el modelo de reflexión de Phong, en el cual se trabaja con los siguientes elementos:
\begin{itemize}
    \item Luz ambiente $i_a$.
    \item Materiales definidos por las constantes $(k_s, k_d, k_a, \alpha)$, que representan los factores de reflexión especular, difusa y ambiente y el coeficiente de brillo respectivamente.
    \item Lista de fuentes de luz $(l_1, \dots, l_m, \dots, l_n)$
    \item Los vectores normalizados definidos para cada punto de la superficie $p$:
          \begin{itemize}
              \item $L_m$: vector director desde $p$ a cada fuente de luz $l_m$.
              \item $N$ vector normal a la superficie en $p$.
              \item $R_m$ dirección del rayo de luz reflejado especularmente desde la fuente $l_m$ en el punto $p$.
              \item $V$: dirección de $p$ a la posición del observador.
          \end{itemize}
\end{itemize}

Con estas variables, el color en $p$ vendrá dado por la fórmula:
\begin{equation*}
    I_p = k_a i_a + \sum_{m=0}^{n} \left(k_d\left(L_m\cdot N\right) + k_s\left(R_m\cdot V\right)^{\alpha} \right)
\end{equation*}

De los vectores anteriores, $L_m$ y $V$ se calculan trivialmente, y $R_m$ se puede obtener como
\begin{equation*}
    R_m = 2(L_m\cdot N)N - L_m
\end{equation*}

Sin embargo, dado que la superficie viene dada por un SDF, no podemos calcular $N$ de forma analítica.

\begin{definicion}
    El \textbf{gradiente} de una función implícita $\phi$ es $\nabla\phi = \left(\frac{\partial \phi}{\partial x}, \frac{\partial \phi}{\partial y}, \frac{\partial \phi}{\partial z}\right)$
\end{definicion}

\begin{proposicion}
    $\nabla\phi$ es perpendicular al isocontorno de $\phi$.
\end{proposicion}
\begin{proof}
    TODO
\end{proof}

\begin{itemize}
    \item $k_s$: factor de reflexión especular.
    \item $k_d$: factor de reflexión difusa.
    \item $k_a$: factor de reflexi
\end{itemize}
\section{Operadores}
\subsection{Transformaciones afines}
\subsection{Booleanos}
\subsection{Fusiones}
\subsection{Deformaciones}