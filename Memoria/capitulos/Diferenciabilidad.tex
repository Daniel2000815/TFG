Antes de seguir avanzando vamos a realizar un estudio de la diferenciabilidad de las funciones distancia con signo, pues nos será de utilidad en las siguientes secciones. Empezamos recordando varios conceptos de análisis diferencial \cite{diff} fijadas las variables $\{x_1,x_2,x_3\}$ y la base usual $$B=\{e_1,e_2,e_3\} = \{(1,0,0),(0,1,0),(0,0,1)\}.$$

Cuando sea conveniente identificaremos
\begin{equation*}
    x_1=x,\quad x_2=y,\quad x_3 = z.
\end{equation*}

\begin{definicion}\label{def:parcial}
    Sea $U$ un abierto de $\R^3$ y $\phi: U \to \R$. Para $i\in \{1,2,3\}$, definimos la \textbf{$\boldsymbol{i}$-ésima derivada parcial} de $\phi$ en $p_0\in\R^3$ como
    \begin{equation*}
        \frac{\partial \phi}{\partial x_i}(p_0) = \lim_{h\to 0}\frac{\phi(p_0+he_i) - \phi(p_0)}{h}.
    \end{equation*}
\end{definicion}

\begin{definicion}
    Sea $U$ un abierto de $\R^3$ y $\phi: U \to \R$. Diremos que $\phi$ es \textbf{diferenciable} en $p_0 \in U$ si existen todas sus derivadas parciales en $p_0$ y son continuas. Definimos la \textbf{diferencial} de $\phi$ en $p_0$ como la suma de todas sus parciales en dicho punto, y la denotamos como $d\phi(p_0)$.

    % \begin{equation*}
    %     \text{lím}_{p\to p_0} \frac{\phi(p)-\phi(p_0) - \nabla\phi(p_0)\cdot(p-p_0)}{\Vert p-p_0\Vert} = 0.
    % \end{equation*}
\end{definicion}

\begin{definicion}
    Dado un abierto $U\subseteq \R^3$, diremos que la \textbf{clase de diferenciabilidad} de una función $\phi:U\to \R$ es $\mathbb{C}^n(U)$ si para $i\in \{1,2,3\}$ y $j\in \{0,\dots, n\}$ existen y son continuas todas las parciales
    \begin{equation*}
        \frac{\partial^j \phi}{\partial x_i}(p),\ \text{para todo } p\in U.
    \end{equation*}
\end{definicion}

\begin{definicion}
  Llamamos \textbf{gradiente} de $\phi\colon \R^3\to\R$ a la función
  \begin{align*}
      \nabla\phi\colon \R^3 &\to \R^3,\\
      p &\mapsto \left(\frac{\partial \phi}{\partial x_1}(p), \frac{\partial \phi}{\partial x_2}(p), \frac{\partial \phi}{\partial x_3}(p)\right).
  \end{align*}
\end{definicion}

\begin{definicion}
  Dada $\phi:\R^3\to \R$ diferenciable, definimos la \textbf{derivada direccional} en $p_0\in \R^3$ en la dirección $v\in\R^3$ a:
  \begin{equation*}
    \nabla_v \phi(p_0) = \nabla \phi(p_0) \cdot v = \frac{\partial{\phi}(p_0)}{\partial{x}}v_x + \frac{\partial{\phi}(p_0)}{\partial{y}}v_y + \frac{\partial{\phi}(p_0)}{\partial{z}}v_z.
  \end{equation*}
\end{definicion}




Ahora mismo, dado una función distancia con signo arbitraria no tenemos información alguna sobre su diferenciabilidad, ya que su expresión puede ser de lo más variada y compleja. Veamos una propiedad que cumplen todos las funciones distancia con signo y que nos permitirá obtener algo de información al respecto \cite{lips,derivWiki}.

\begin{definicion}
    Una campo escalar $\phi\colon \R^3 \to \R$ se dice \textbf{lipschitziano} si existe una constante $L>0$ tal que
    \begin{equation*}
        \vert \phi(p)-\phi(q)\vert \le L\Vert p-q\Vert,\ \text{para todo } p,q\in \R^3.
    \end{equation*}
    La constante $L$ recibe el nombre de \textbf{constante de Lipschitz}.
\end{definicion}

\begin{proposicion}
    Sea $\phi\colon \R^3 \to \R$ una función lipschitziana cualquiera con constante de Lipschitz $L$. Entonces
    \begin{equation*}
        \vert d\phi(p)\vert \le L
    \end{equation*}
    en todo punto donde sea diferenciable.
\end{proposicion}

\begin{lema}
    Sea $\phi:\R^3\to \R$ la función distancia con signo asociada a $\Omega$. Entonces $\phi$ es lipschitziana con constante $L=1$.
\end{lema}
\begin{proof}
    Sean $p$ y $q\in \R^3$. Usando la \autoref{def:sdf}, para todo $s\in \delta\Omega$ se tiene
    \begin{equation*}
        \phi(p) \le \Vert p-s\Vert = \Vert p-q+q+s\Vert \le \Vert p-q\Vert + \Vert q-s\Vert.
    \end{equation*}
    Por tanto, $\phi_{\Omega}(p) - \Vert p-q\Vert \le \Vert q-s \Vert$, luego $\phi_{\Omega}(p) - \Vert p-q\Vert \le \Inf_{s\in \delta\Omega}(\Vert q-s \Vert) = \phi_{\Omega}(q)$ y obtenemos
    \begin{equation*}
         \phi_{\Omega}(p) - \phi_{\Omega}(q) \le \Vert p-q \Vert.
    \end{equation*}
    De forma análoga podemos ver que $\phi_{\Omega}(q) - \phi_{\Omega}(p) \le \Vert q-p \Vert$, concluyendo que
    \begin{equation*}
        \vert \phi_{\Omega}(p) - \phi_{\Omega}(q)\vert \le 1\cdot \Vert p-q\Vert.\qedhere
    \end{equation*}
\end{proof}

\begin{lema}[Teorema de Rademacher]
    Sea $U$ un abierto de $\R^3$ y $\phi:U\to \R$ lipschitziana. Entonces $\phi$ es diferenciable en casi todo punto de $U$.
\end{lema}

Tenemos por tanto asegurado que $\phi_{\Omega}$ será diferenciable en casi todo punto de $\R^3$. No obstante, podemos concretar aún más dónde están los puntos de conflicto cuando $\Omega$ sea lo suficientemente regular \cite{dif1,dif2}. Para ello necesitaremos introducir el concepto de esqueleto de una superficie \cite{derivWiki} y repasar algunas definiciones básicas asociadas a superficies en el espacio \cite{apuntes:curvas}.
\begin{definicion}
    Sea $\phi_{\Omega} \colon \R^3\to \R$ una función distancia con signo. Llamamos \textbf{esqueleto} de $\Omega$ al conjunto de puntos de $\R^3$ cuya distancia a la superficie puede obtenerse como la distancia a dos o más puntos distintos de $\delta \Omega$:
    \begin{equation}
        \epsilon(\Omega) = \{p\in \R^3 : \phi_{\Omega}(p) = \Vert p-q\Vert = \Vert p-r\Vert,\ q,r\in \delta\Omega ,\ q\neq r \}.
    \end{equation}
\end{definicion}

\begin{definicion}
    Dado $I\subseteq \R$, llamamos \textbf{curva parametrizada} a una aplicación
    \begin{align*}
        \alpha: I &\to \R^3,\\
        t &\mapsto (x(t), y(t), z(t)),
    \end{align*}
    donde $x,y,z:I\to \R$ son diferenciables. 
\end{definicion}
\begin{definicion}
    Decimos que $\Omega \subseteq \R^3$ es una \textbf{superficie regular} si para cada $p\in \Omega$ existen abiertos $U\subseteq \R^2$ y $V\subseteq \R^3$ junto a una aplicación $\psi\colon U \to V\cap \Omega$ tal que:
    \begin{enumerate}
        \item $\psi$ es un homeomorfismo, es decir, es continua, biyectiva y con inversa continua,
        \item $\psi$ es diferenciable y su diferencial es inyectiva.
    \end{enumerate}
\end{definicion}

\begin{definicion}
    Sea $\Omega$ una superficie regular y $p\in \Omega$. Dados $\varepsilon>0$ y una curva parametrizada diferenciable
    $$\alpha: ]\varepsilon, \varepsilon[ \to \R^3 \text{ tal que } \text{Img($\alpha$)} \subset \Omega \text{ y } \alpha(0)=p,$$
    diremos que $\alpha'(0)$ es un \textbf{vector tangente} a $\Omega$ en $p$.
\end{definicion}
\begin{definicion}
    Sea $\Omega$ una superficie regular, $p\in \Omega$ y $T_p\Omega$ el plano vectorial conteniendo todos los vectores tangentes a $\Omega$ en $p$. Llamamos \textbf{plano tangente} a $\Omega$ en $p$ al conjunto $p+T_p\Omega$.
\end{definicion}
\begin{definicion}
    Sea $\Omega$ una superficie regular y $p\in \Omega$. El \textbf{vector normal} a $\Omega$ en $p$ es el vector $N_p\in \R^3$ de norma uno perpendicular al plano tangente de $\Omega$ en $p$.
\end{definicion}
% \begin{definicion}
%     Sea $\Omega \subset \R^3$ y $p\in \Omega$. Llamamos \textbf{vector normal} en $p$ al vector de norma uno y perpendicular al borde de $\Omega$ en $p$. Lo denotamos $N_p$.
% \end{definicion}



El siguiente teorema, cuya demostración podemos consultar en \cite{dif1}, nos proporciona una caracterización geométrica de la diferenciabilidad de cualquier función distancia con signo $\phi_{\Omega}$ bajo ciertas hipótesis de regularidad para $\Omega$.
\begin{teorema}\label{teo:diff}
    Sea $\Omega \subseteq \R^3$ cuya frontera es regular y $\phi_{\Omega} \colon \R^3\to \R$ la función distancia con signo asociada a $\Omega$. Entonces $\phi_{\Omega}$ es diferenciable en un entorno tubular $U$ de $\delta \Omega$. Es más, para cada $p\in \R^3$ se cumple una se las siguientes propiedades:
    \begin{enumerate}
        \item $p\in \delta \Omega$ y $\phi_{\Omega}$ es diferenciable en $p$ con $\nabla S_{\phi(p)} = N_p$,
        \item $p\notin \delta \Omega$ y $\phi_{\Omega}$ es diferenciable en $p$ si y solo si $p\in \R^3\setminus \epsilon(\Omega)$, en cuyo caso
        \begin{equation*}
            \nabla \phi_{\Omega}(p) = \frac{q-p}{\phi_{\Omega}(p)},
        \end{equation*}
        donde $q$ es el único punto de $\delta\Omega$ tal que $\phi_{\Omega}(p) = \Vert q-p\Vert$.
    \end{enumerate}
\end{teorema}

\begin{corolario}
    Toda función distancia con signo $\phi\colon \R^3\to \R$ satisface la ecuación de la eikonal 
    \begin{equation*}
        \Vert \nabla \phi(p)\Vert = 1
    \end{equation*}
    en todo punto $p$ donde sea diferenciable.
\end{corolario}
% A la vista de este resultado podemos diferenciar $\phi$ con tranquilidad, pues solo estamos interesados en estudiar el gradiente en puntos de $\delta\Omega$, luego
% \begin{equation*}
    
% \end{equation*}

% TODO: ver si decir que podria calcularse fuera del shader pasanndo el string al shader y recompilando. Si no, ver si comentar algo de que hay que calcularlo cada frame, y por eso no podemos hacerlo analiticamente
% De ser derivable, podría ser una buena opción calcular el gradiente una única vez al momento de definir $\phi$, tras lo cual para obtener $N$ solo habría que realizar evaluaciones de dicho gradiente. Sin embargo, esto requeriría de varias comprobaciones previas, y aún necesitaríamos otro método para tratar las no diferenciables. Por ello nos decantaremos por un método numérico más sencillo que únicamente hará uso de evaluaciones de $\phi$.

