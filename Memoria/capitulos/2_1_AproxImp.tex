Empezábamos el capítulo diciendo que una de las representaciones más comunes de superficies es a través de implícitas, pero hasta ahora nos hemos centrado en estudiar un tipo particular de este tipo de ecuaciones. Si intentásemos aplicar los algoritmos anteriores a una función implícita cualquiera, observaríamos que el resultado tiene ciertos defectos, tales como deformaciones, o incluso no se visualiza. Vamos a introducir una técnica que dada una función implícita $\phi$, nos permitirá obtener un SDF aproximado de $S_\phi$. Esto nos será útil cuando no conozcamos o no podamos calcular explícitamente el SDF de una superficie pero sí su representación implícita.

\begin{proposicion}
    Sea $\phi\colon \R^3\to\R$ una función implícita cualquiera y $p\in\R^3$. Entonces: 
    \begin{equation*}    
        \sdf_{S_\phi}(p) \approx \frac{\vert \phi(p)\vert}{\vert \nabla\phi(p)\vert}
    \end{equation*}
\end{proposicion}
\begin{proof}
    Sea $q$ al punto de $S_\phi$ más cercano a $p$ y $v=\vec{pq} \perp S_\phi$ en $q$. Queremos calcular la distancia de $p$ a $S_\phi$, que es justamente $v$. Podemos calcular el desarrollo de Taylor de $\phi$ en $p$:
    \begin{equation*}
        \phi(p+v) = \phi(p) + \nabla(p)\cdot (p+v-p) + \mathcal{O}(\vert p+v-p\vert^2) \implies \phi(p+v) = \phi(p) + \nabla(p)\cdot v + \mathcal{O}(\vert v\vert^2)
    \end{equation*}

    Suponemos ahora que $p$ está cerca de $S_\phi$, esto es, $\vert v\vert < \varepsilon$. Como $\phi(p+v) = \phi(q)=0$, tenemos que:
    \begin{align*}
        0 &= \vert \phi(p+v)\vert \approx \vert \phi(p) + \nabla(p)\cdot v \vert \le \vert \phi(p)\vert - \vert \nabla(p)\cdot v \vert\\
         &\le \vert \phi(p)\vert - \vert \nabla(p)\vert \vert v \vert \implies \vert v\vert \le \frac{\vert \phi(p)\vert}{\vert \nabla(p)\vert}
    \end{align*}
    donde hemos usado la desigualdad triangular y las propiedades del producto escalar.
\end{proof}

Al igual que nos ocurría al calcular el vector normal, habrá ocasiones en las que el gradiente no pueda ser obtenido de forma analítica. Esta vez sin embargo no nos basta con conocer solo la dirección del gradiente, así que no podemos aplicar de nuevo el método del tetraedro. Usaremos un nuevo método en su lugar que usa la definición de derivada.

\begin{proposicion}[Método de las diferencias centrales]
    Sea $\phi\colon \R^3\to\R$. Entonces:
    \begin{align*}
        \frac{\partial\phi}{\partial x}(p) &\approx \frac{\phi(p+(h,0,0)) - \phi(p-(h,0,0))}{2h}\\[10pt]
        \frac{\partial\phi}{\partial y}(p) &\approx \frac{\phi(p+(0,h,0)) - \phi(p-(0,h,0))}{2h}\\[10pt]
        \frac{\partial\phi}{\partial z}(p) &\approx \frac{\phi(p+(0,0,h)) - \phi(p-(0,0,h))}{2h}
    \end{align*}
    
    donde $h\in\R^+$ es un valor arbitrariamente pequeño. 
\end{proposicion}

Para calcular el gradiente necesitaremos evaluar $\phi$ un total de seis veces, ya que hay que realizar dos evaluaciones por cada variable. Es por este motivo que 
