Empezábamos el capítulo diciendo que una de las representaciones más comunes de una superficie es a través de implícitas, pero hasta ahora nos hemos centrado en estudiar un subconjunto de esta familia. Si intentásemos aplicar el algoritmo de \textit{spheretracing} a una función implícita cualquiera podríamos observar que el resultado presenta defectos, tales como deformaciones o grietas, o que incluso no se visualiza. Veamos qué podemos hacer para, dada una función implícita $\phi$, obtener información aproximada de $S_\phi$ \cite{article:aprox}. Esto nos será útil cuando no conozcamos o no podamos calcular explícitamente el SDF de una superficie, pero sí su representación implícita.

\begin{proposicion}
    Sea $\phi\colon \R^3\to\R$ una función implícita cualquiera. Entonces
    \begin{equation*}    
        \vert \sdf_{S_\phi}(p)\vert \le \frac{\vert \phi(p)\vert}{\Vert \nabla\phi(p)\Vert}.
    \end{equation*}
\end{proposicion}
\begin{proof}
    Fijamos el punto $p$ del cual queremos aproximar la distancia a $S_{\phi}$. Sea $q$ el punto de $S_\phi$ más cercano a $p$ y $v=\vec{pq}$. Queremos calcular la distancia de $p$ a $S_\phi$, que será justamente $\Vert v\Vert$. Podemos realizar el desarrollo de Taylor de $\phi$ centrado en $p$:
    \begin{equation*}
        \phi(p+v) = \phi(p) + \nabla(p)\cdot (p+v -p) + \mathcal{O}(\vert p+v-p)\vert^2) = \phi(p) + \nabla(p)\cdot v + \mathcal{O}(\vert v\vert^2).
    \end{equation*}

    Suponemos ahora que $p$ está cerca de $S_\phi$, esto es, $\exists \varepsilon>0$ tal que  $\vert v\vert < \varepsilon$, de forma que podemos obviar el residuo. Como $\phi(p+v) = \phi(q)=0$, tenemos que
    \begin{align*}
        0 &= \vert \phi(p+v)\vert \le \vert \phi(p) + \nabla(p)\cdot v \vert \le \vert \phi(p)\vert - \vert \nabla(p)\cdot v \vert\\
         &\le \vert \phi(p)\vert - \Vert \nabla(p)\Vert\cdot \Vert v \Vert \implies \Vert v\Vert \le \frac{\vert \phi(p)\vert}{\Vert \nabla(p)\Vert},
    \end{align*}
    donde hemos usado la desigualdad triangular y la linealidad del producto escalar.
\end{proof}
\begin{corolario}
    Sea $\phi\colon \R^3\to \R$ lipschitziana de constante $L$. Entonces
    \begin{equation*}    
        \vert \sdf_{S_\phi}(p)\vert \le \frac{\vert \phi(p)\vert}{\Vert L\Vert}.
    \end{equation*}
\end{corolario}

Este resultado solo nos proporciona una cota superior de la función distancia (sin signo). En nuestro caso esto es suficiente, pues esta nos sigue permitiendo representar la frontera de $S_{\phi}$. En su artículo \cite{art:impSdf}, Pierre-Alain Fayolle describe un método para obtener un SDF asociado a una superficie implícita que representa de manera exacta su frontera. Para ello descompone el SDF como 
\begin{equation*}
    sdf_{S_{\phi}}(p;\theta) = \phi(p)g(p;\theta)\quad \text{ o }\quad sdf_{S_{\phi}}(p;\theta) = sign(\phi(p))g(p;\theta),
\end{equation*}

donde $g$ es una función paramétrica de parámetros $\theta$ y $sign$ es una versión suavizada de la función signo, por ejemplo $sign(x) = \tanh{(kx)},\ k\in\R$. Para obtener la expresión de $g$ introduce la función implícita $\phi$ en la capa final de una red neuronal entrenada para  minimizar una función pérdida asociada al SDF, y para ajustar $\theta$ expresa $sdf_{S_{\phi}}(p;\theta)$ como la solución de un problema variacional. No obstante, esta técnica está fuera del ámbito de este trabajo, de forma que nos limitaremos a obtener la cota de la función distancia.