% !TeX root = ../libro.tex
% !TeX encoding = utf8

Durante todo el capítulo fijamos $A$ anillo conmutativo y $X=\{x_1, \dots, x_n\}$ conjunto de variables distintas.

\section{Polinomios en varias variables}

\begin{definicion}
  Llamamos \textbf{monomio} en $X$ a cualquier producto de la forma:
  $$X^{\alpha} = x_1^{\alpha_1} \cdots x_n^{\alpha_n}\quad, \alpha_i \in \N\ \forall 0\le i \le n$$
  Lo denotaremos como $X^{\alpha}$, con $\alpha\in \N^n$.
\end{definicion}

\begin{definicion}
  Definimos un \textbf{polinomio multivariable} a cualquier combinación lineal finita de monomios $\sum_{\alpha\in\Nn} a_{\alpha} X^{\alpha}$. Puede comprobarse que el conjunto de polinomios forma un cuerpo, que denotaremos $A[X] = A[x_1,\dots, x_n]$.
\end{definicion}

Tras esta definición nos surge la pregunta de si hay alguna forma "natural" de ordenar los monomios de un polinomio. Vamos a ver que no hay una única forma, sino que hay muchas igual de válidas.

\begin{definicion}
  Un \textbf{orden admisible} es un orden total $\le$ sobre $\N^n$ cumpliendo:
  \begin{enumerate}
    \item $(0,\dots ,0) \le \alpha,\ \forall \alpha \in \N^n$.
    \item $\alpha < \beta \implies \alpha + \gamma < \beta + \gamma,\ \forall \alpha,\beta,\gamma \in \N^n$.
  \end{enumerate}
\end{definicion}

Hay muchos órdenes admisibles, pero nosotros usaremos uno en particular:

\begin{definicion}
  Definimos el \textbf{orden lexicográfico} $\le_{\text{lex}}$ como:
  \begin{equation*}
    \alpha \lex \beta \iff \begin{cases}
      \alpha  = \beta \\
      \quad\text{ó}   \\
      \alpha_i < \beta_i \text{, donde $i$ es el primer índice tal que } \alpha_i \neq \beta_i
    \end{cases}
  \end{equation*}
\end{definicion}

Ahora ya estamos en disposición de definir varios conceptos que nos resultarán imprescindibles:

\begin{definicion}
  Sea $f= \sum_{\alpha\in \Nn} a_{\alpha} X^{\alpha}$ y $\le$ admisible. Definimos:
  \begin{itemize}
    \item \textbf{Exponente:} $exp(f) = \max_{\le}(\alpha)$.
    \item \textbf{Monomio líder:} $\lmf = X^{\expf}$
    \item \textbf{Coeficiente líder:} $\lcf = a_{\expf}$.
    \item \textbf{Término líder:} $\ltf = \lcf \cdot \lmf$.
  \end{itemize}
\end{definicion}

A partir de ahora siempre que usemos un orden supondremos que es admisible, y si no se especifica, que es $\lex$.

Estudiamos ahora las operaciones que podemos realizar sobre $A[X]$.

\begin{proposicion}
  Sean $f,g\in A[X]$.
  \begin{itemize}
    \item $(f+g)(\alpha) = f(\alpha) + f(\alpha)$.
    \item $(fg)(\alpha) = \sum_{\beta+\gamma=\alpha} f(\beta)g(\gamma)$.
  \end{itemize}
\end{proposicion}

\begin{teorema}
  Sea $F=\{f_1,\dots, f_s\} \subset A[X]$. Entonces todo polinomio $f\in A[X]$ se puede expresar como:
  \begin{equation*}
    f = q_1f_1 + \cdots + q_sf_s + r
  \end{equation*}

  donde $q_i, r\in A[X]$ y $r$ no se puede expresar como combinación lineal de ningún elemento de $F$ o es $0$. Notaremos $\rem(f, [F]) = r$.
\end{teorema}

En otras palabras, podemos dividir $f$ entre los polinomios $f_1, \dots, f_s$. Para realizar la demostración primero presentamos un algoritmo que dados $f,f_1,\dots, f_s \in A[X]$, expresa $f$ de la forma vista en el teorema, para después comprobar que funciona correctamente siempre.

\SetKwComment{Comment}{/* }{ */}
\begin{algorithm}[hbt!]
  \caption{División polinomios varias variables}\label{alg:two}
  \KwData{$f$, $F = \left[ f_1, \dots, f_S\right]$}
  $p\gets f$\;
  $\left[q_1,\dots, q_s\right] \gets \left[0,\dots, 0\right]$\;
  $r\gets 0$\;

  \While{$p \neq 0$}{
    $\text{divisorEncontrado} \gets false$\;
    \For{$f_i \in F$} {
      \If{$\text{exp}(p) = \text{exp}(f_i) + \alpha$}{
        $q_i\gets q_i + \frac{\text{lc}(p)}{\text{lc}(f_i)} X^{\alpha}$\;
        $p \gets p - f_i \cdot \frac{\text{lc}(p)}{\text{lc}(f_i)} X^{\alpha}$\;
        $\text{divisorEncontrado} \gets true$\;
      }
    }
    \If{$!\text{divisorEncontrado}$}{
      $r \gets r + \text{lt}(p)$\;
      $p \gets p - \text{lt}(p)$\;
    }
  }
  \Return{$\left[r,q_1,\dots, q_s\right]$}
\end{algorithm}

\begin{proof}
  TODO
\end{proof}
\section{Bases de Groebner}

\begin{definicion}
  Dado un anillo conmutativo $R$, decimos que $\emptyset \neq I \subseteq R$ es un \textbf{ideal} si:
  \begin{enumerate}
    \item $a+b\in I,\ \forall a,b\in I$.
    \item $ab\in I,\ \forall a\in I,\ \forall b\in R$.
  \end{enumerate}

  Lo denotaremos como $I\le R$.
\end{definicion}

\begin{definicion}
  Dado $F\subseteq A$, el \textbf{ideal generado} por $F$ es:
  \begin{equation*}
    \langle F \rangle = \{a_1f_1 + \cdots + a_sf_s : a_1,\dots, a_s\in A,\ f_1,\dots, f_s\in F\}
  \end{equation*}
\end{definicion}

\begin{definicion}
  Dado $I\le A[X]$, diremos que $G = \{g_1,\dots, g_s\}\subseteq I$ es una \textbf{base de Groebner} para $I$ si $\langle \lt(I)\rangle = \langle \lt(g_1),\dots, \lt(g_t) \rangle$.
\end{definicion}

\begin{definicion}
  Dada $f\in A[X]$, definimos su \textbf{soporte} como $$\supp(f) = \{ \alpha\in\Nn : f(\alpha) \neq 0\}$$
\end{definicion}

\begin{definicion}
  $G$ se dirá una \textbf{base de Groebner reducida} para $I$ si para todo $g\in G$ se cumple:
  \begin{itemize}
    \item $\lcg=1$.
    \item $\supp(g) \cap \left(\exp(G\setminus\{g\}) + \Nn \right) = \emptyset$.
  \end{itemize}
\end{definicion}

La base de Groebner es a un ideal lo que un sistema de generadores a un espacio vectorial: un subconjunto a partir del cual podemos obtener el total. Si ademças la base es reducida, esta será el equivalente a la base del espacio vectorial. Por tanto, resulta de gran interés saber si dado un ideal $I$ siempre existirá una base de Groebner asociada, y de ser así, si podemos calcularla.

\begin{definicion}
  Dados $\alpha,\beta \in \Nn$, definimos su \textbf{mínimo común múltiplo} como
  \begin{equation*}
    \lcm(\alpha,\beta) = \{\max(\alpha_1, \beta_1),\dots, \max(\alpha_s, \beta_s)\}
  \end{equation*}
\end{definicion}

\begin{definicion}
  Sean $f,g \in A[X]$ donde $\alpha=\exp(f),\ \beta=\exp(g),\ \gamma = \lcm(\alpha,\beta)$. Definimos el \textbf{S-polinomio} de $f$ y $g$ como
  \begin{equation*}
    S(f,g) = \lc(g)X^{\gamma-\alpha}f - \lc(f)X^{\gamma-\beta}g
  \end{equation*}
\end{definicion}

\begin{teorema}[Criterios de Buchberger]
  Sean $I\le A[X]$ y $G=\{g_1,\dots, g_t\}$ conjunto de generadores de $I$. Entonces:
  \begin{enumerate}
    \item $G \text{ es base de Groebner para } I \iff \rem(S(g_i,g_j), G)=0,\ \forall 1\le i<j\le t$.
    \item $\lcm(\exp(g_i),\exp(g_j)) = \exp(g_i)\exp(g_j) \implies \rem(S(g_i,g_j), G) = 0$.
    \item $\exists g_t\neq g_i,g_j,\ \alpha\in \N^n : \lcm(\exp(g_i), \exp(g_j)) = \alpha \exp(g_t) \implies \rem(S(g_i,g_j), G)$ si $S(g_i,g_t)$ y $S(g_j,g_t)$ ya han sido considerados.
  \end{enumerate}
\end{teorema}

El algoritmo del cálculo de la base de Groebner se basará en el primer criterio, mientras que los otros nos servirán para evitar cálculos innecesarios de S-polinomios.

\SetKwComment{Comment}{/* }{ */}
\begin{algorithm}[hbt!]
  \caption{Cálculo de base de Groebner reducida}\label{alg:div}
  \KwData{$f$, $F = \left[ f_1, \dots, f_S\right]$}

  $G\gets F$\;

  \Repeat{$G' = G$}{
    $G'\gets G$\;
    \For{each pair $\{f,g\} \subseteq G'$} {
      \If{$\text{Criterio 2}(f,g) \textbf{ AND } !\text{Criterio 3}(\{f,g\}, G, G')$}{
        $r\gets \rem(S(f,g), G')$\;
        \If{$r\neq 0$}{
          $G\gets G\cup \{r\}$\;
        }
      }
    }
  }

  \Return{$G$}
\end{algorithm}

\section{Implicitación Polinomial}

\begin{definicion}Dado $F \subseteq A[X]$, llamamos \textbf{variedad afín} definida por $F$ al conjunto:
  \begin{equation*}
    \mathbb{V}(F) = \{(a_1,\dots, a_n)\in A^n : f_i(a_1,\dots, a_n)=0 \ \forall 1\le i \le s\}
  \end{equation*}
\end{definicion}

\begin{definicion}
  Dado $I\le A[x_1,\dots,x_n]$, diremos que su \textbf{ideal de $l$-eliminación} es:
  \begin{equation*}
    I_l = I \cap A[x_{l+1}, \dots, x_n] \le A[x_{l+1}, \dots, x_n]
  \end{equation*}
\end{definicion}

\begin{teorema}[Eliminación]
  Sea $I\le A[x_1,\dots,x_n]$ y $G$ una base de Groebner suya respecto al orden \textit{lex} con $x_1 > x_2 > \cdots > x_n$. Entonces, una base de Groebner de $I_l$ es:
  \begin{equation*}
    G_l = G\cap A[x_{l+1},\dots, x_n]
  \end{equation*}
\end{teorema}

\begin{teorema}[Implicitación Polinomial]
  Dados $f_1,\dots, f_n \in A[t_1, \dots, t_r]$ con $A$ cuerpo infinito, sea
  \begin{equation*}
    \begin{align*}
      \phi \colon A^r  & \to A^n                                                              \\
      (a_1,\dots, a_r) & \mapsto \left(f_1(a_1,\dots, a_r), \dots, f_n(a_1,\dots, a_r)\right)
    \end{align*}
  \end{equation*}

  Tomamos los ideales:
  \begin{itemize}
    \item $I = \langle x_1-f_1,\dots,  x_1-f_1\rangle \le A[t_1,\dots, t_r,x_1\dots, x_n]$
    \item $J = I\cap A[x_1,\dots, x_n]$ el ideal de $r$-eliminación de $I$
  \end{itemize}

  Entonces, $\mathbb{V}(J)$ es la menor variedad que contiene a $\phi(A^r)$.
\end{teorema}
