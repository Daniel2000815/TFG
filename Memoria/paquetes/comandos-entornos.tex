% DEFINICIÓN DE COMANDOS Y ENTORNOS

% CONJUNTOS DE NÚMEROS

\newcommand{\N}{\mathbb{N}}     % Naturales
\newcommand{\R}{\mathbb{R}}     % Reales
\newcommand{\Z}{\mathbb{Z}}     % Enteros
\newcommand{\Q}{\mathbb{Q}}     % Racionales
\newcommand{\C}{\mathbb{C}}     % Complejos
\newcommand{\Nn}{\mathbb{N}^n}
\newcommand{\lex}{\le_{\text{lex}}}
\newcommand{\expf}{\text{exp}(f)}
\newcommand{\lcf}{\text{lc}(f)}
\newcommand{\lc}{\text{lc}}
\newcommand{\lcg}{\text{lc}(g)}
\newcommand{\lmf}{\text{lm}(f)}
\newcommand{\ltf}{\text{lt}(f)}
\newcommand{\lt}{\text{lt}}
\newcommand{\rem}{\text{rem}}
\newcommand{\lcm}{\text{lcm}}

% TEOREMAS Y ENTORNOS ASOCIADOS

% \newtheorem{theorem}{Theorem}[chapter]
\newtheorem*{teorema*}{Teorema}
\newtheorem{teorema}{Teorema}[chapter]
\newtheorem{proposicion}{Proposición}[chapter]
\newtheorem{lema}{Lema}[chapter]
\newtheorem{corolario}{Corolario}[chapter]

\theoremstyle{definition}
\newtheorem{definicion}{Definición}[chapter]
\newtheorem{ejemplo}{Ejemplo}[chapter]

\theoremstyle{remark}
\newtheorem{observacion}{Observación}[chapter]
