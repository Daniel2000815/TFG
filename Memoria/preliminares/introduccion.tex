% !TeX root = ../libro.tex
% !TeX encoding = utf8
%
%*******************************************************
% Introducción
%*******************************************************

% \manualmark
% \markboth{\textsc{Introducción}}{\textsc{Introducción}} 

\chapter{Introducción}
En el campo de los gráficos generados por computador los métodos más asentados para describir objetos tridimensionales han sido mediante polígonos o ecuaciones implícitas. Sin embargo, en los últimos años ha surgido un enfoque alternativo que muestra un gran potencial: la utilización de funciones distancia con signo (SDF). Estos permiten representar objetos geométricos mediante una única función escalar que asigna a cada punto del espacio su distancia con signo respecto a la superficie del objeto. Veremos que esta representación es especialmente útil para la generación de imágenes y el renderizado en tiempo real, ofreciendo ventajas significativas en términos de eficiencia y precisión respecto otros métodos tradicionales.\newline


El objetivo principal de este Trabajo de Fin de Grado (TFG) es la creación de una aplicación web interactiva que facilite la creación e interacción con superficies generadas a través de SDF de forma intuitiva para el usuario sin que este necesite conocimientos al respecto. Para lograr este objetivo, llevaremos a cabo el desarrollo de la aplicación aprovechando las tecnologías web y \textit{frameworks} modernos, evaluando la eficiencia y el rendimiento de la aplicación en términos de tiempos de respuesta y calidad visual.\newline


Empezaremos este trabajo presentando los SDF explorando los fundamentos teóricos y matemáticos detrás de los SDF y comprendiendo su capacidad para describir superficies y volúmenes de manera precisa y compacta, así como las técnicas empleados para su manipulación, como las operaciones booleanas o las deformaciones. Posteriormente estudiaremos cómo pueden ser representados mediante \textit{spheretracing} y aplicando técnicas avanzadas como \textit{antialiasing} y oclusión ambiental .\newline

Una vez comprendidos los detalles de los SDF nos centraremos en estudiar como aprovechar sus ventajas para superficies dadas de forma implícita o paramétrica obteniendo una representación suya como SDF usando bases de Groebner

\endinput
