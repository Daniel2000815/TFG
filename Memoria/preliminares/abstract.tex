\thispagestyle{empty}

\begin{center}
{\large\bfseries \miTitulo: \miSubtitulo}\\
\end{center}
\begin{center}
\miNombre (alumno)\\
\end{center}

%\vspace{0.7cm}
\noindent{\textbf{Palabras clave}: función distancia con signo, \textit{raytracing}, \textit{spheretracing}, modelo \textit{BlobTree}, polinomios en varias variables, bases de Gröbner, implicitación.}\\

\vspace{0.7cm}
\noindent{\textbf{Resumen}}\\

Buscar maneras de representar información de forma más eficiente o que nos permitan aplicar nuevas técnicas es una constante tanto en el ámbito de la informática como en el de las matemáticas. En este trabajo estudiaremos las funciones distancia con signo, que miden la distancia de cada punto a un conjunto según este se encuentre a un \qq{lado} u otro del conjunto, y veremos como usarlas para representar superficies en tiempo real a través de \textit{raytracing} y manipularlas usando el modelo en forma de árbol \textit{BlobTree}. Lejos de quedarnos ahí, veremos como obtener una función distancia con signo aproximada a partir de superficies definidas a través de ecuaciones implícitas o paramétricas, habiendo explorado previamente el problema de implicitación y su solución con bases de Gröbner para estas últimas.\newline

El objetivo de este trabajo es mostrar la utilidad de esta forma de representar geometría, implementando una aplicación web que lleve a la práctica de manera eficiente todos los conceptos y técnicas asociados a las funciones distancia con signo y bases de Gröbner que veremos a lo largo del trabajo. En particular, la aplicación usará la API de rasterización para permitir que el algoritmo de \textit{raytracing} pueda ejecutarse en la mayor parte de dispositivos actuales. También se desarrollará una librería de código abierto propia que será usada por la aplicación para trabajar con polinomios en varias variables y resolver el problema de implicitación.
\cleardoublepage


\thispagestyle{empty}


\begin{center}
{\large\bfseries \miTitle: \miSubtitle}\\
\end{center}
\begin{center}
\miNombre (student)\\
\end{center}

%\vspace{0.7cm}
\noindent{\textbf{Keywords}: Keyword1, Keyword2, Keyword3, ....}\\

\vspace{0.7cm}
\noindent{\textbf{Abstract}}\\

Write here the abstract in English.

\thispagestyle{empty}
\cleardoublepage
\endinput
